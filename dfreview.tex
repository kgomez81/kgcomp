\documentclass[12pt]{article}
\usepackage{amsmath}
\usepackage{amssymb}
\usepackage[margin=1.0in]{geometry}


\title{A Review of M. M. Desai's and D. S. Fisher's "Beneficial Mutation-Selection Balance and the Effect of Linkage on Positive Selection"}
\date{Fall 2016}
\author{Kevin Gomez}

\begin{document}
\maketitle
\newpage

%Writing Resources
%https://academicskills.anu.edu.au/node/492

\newpage
\subsection*{Introduction}
Our understanding of beneficial mutations and their dynamics is comparatively less developed than it is for neutral and deleterious mutations.  The detailed work carried out for the latter two has provided important insights into genetic diversity in populations and led to development of important tools to study and characterize key evolutionary processes in species.  However, beneficial mutations play a much more fundamental role in the long-term adaptation of species, and they also greatly influence the levels of genetic diversity at linked sites.  Unfortunately, our understanding of them limited to the simplest cases due to the complexity often involved in the biological scenarios where positive selection occurs.  This makes the work of Desai and Fisher an important contribution to the existing body of theoretical literature on beneficial mutations, and in particular, their detailed analysis provides a solid framework for modeling adaptation in asexual populations.  

The rate at which beneficial mutations occur determines the dynamics of long-term adaptation, assuming that magnitude of their selective advantage is large enough to dominate drift determined by the population size.  There are two regimes of interest, which have been investigated to some extent in past work.  In the first of these, positive selection is characterized by fairly simple dynamics, which are well understood.  The second regime leads to much more complicated behavior, where prior efforts to properly analyze its dynamics have been incomplete (fix this!).        

The simplest regime, known as the strong-selection weak-mutation regime, occurs when the average time between appearances of beneficial mutations is much larger than the time required for selection to fix a beneficial mutations.  In this particular scenario, selection purges variation in fitness much quicker than it can be introduced into the population, which results in a succession of a single ruling classes composed of identically fit individuals.  For this reason, it is referred to as the successional regime.  

The second is the strong-selection strong-mutation regime, which occurs when the time between appearances of beneficial mutations is comparable to or smaller than time needed for beneficial mutations to fix in the population.  In this particular regime, selection fails to purge any variation in fitness before more is introduced with the appearance of the next beneficial mutation.  It is denoted the concurrent mutations regime due to the fact that these dynamics give rise a population with concurrent clonal subpopulations, referred to as fitness classes, since subpopulations posses distinct fitness levels, and each is composed of individuals with identical fitnesses.  

This second regime is the focus of the paper, starting with the simplest model, followed by an exploration of it's application to more complicated situations associated with the relaxation of certain assumptions.  Desai and Fisher's complete work is extensive and far reaching in its results, but I will limit myself to covering their simplest model, with only a subset of its extensions. 

\subsection*{The Central Idea Behind the Model}
The full analysis presented in Desai and Fisher are preceded by a heuristic analysis that provide insights into the full results later obtained.  This basic analysis starts with the same set of assumptions that are later adopted in the more careful approach taken, and consist of considering asexually reproducing populations with fixed population size $N$, that give rise to beneficial mutations at rate $U_b$ whose selective advantage is constant is $s$.  Additionally, it is assumed that that selection dominates drift requires that $1/N \ll s \ll 1$, and that deleterious mutations do not occur in the population. 

The former of these assumptions provides a separation of stochastic and deterministic behavior of fitness classes based on their size.  For a fitness class of size $n$, it takes random drift at least $n$ generations to decrease its size by $n$.  If the selective advantage of the subpopulation is $s$, then in $n$ generations its size will increase on average by approximately $n(1+s)^n-n \approx n^2s$.  Consequently, if $n^2s \ge n$, or equivalently $n>1/s$, then selection dominates.  Otherwise, drift dominates when $n\lq 1/s$, and it follows from the assumption $1/s \ll N$ that such smaller subpopulations are entirely dominated by drift and can be treated stochastically.

In the successional regime, the one ruling class generates mutants at an average rate of $N U_b$.  Only a proportion of them are able to survive stochastic forces.  This proportion is given by the probability of fixation, which is proportional to $s$ and equal to $s$ in the case of Desai and Fisher's simple model.  Beneficial mutations that are able to overcome drift eventually give rise to subpopulations that grow to roughly the size of $1/s$, where selection then dominates and fixation of the fitness class subsequently follows.  In this scenario the beneficial mutation is said to establish.  It follows that the occurrence of beneficial mutations destined to establish follow a homogeneous Poisson process with mean rate rate $NU_b s$, and furthermore, the average time between establishments, denoted $\langle \tau_{est} \rangle$, is $\langle \tau_{est} \rangle = 1/(N U_b s)$.  Once a beneficial mutation establishes, the corresponding subpopulation grows exponentially at a rate of $s$.  Its growth in time is then approximately $n(t)\approx (1/s) e^{st}$, and implies that the time until fixation will be $\tau_{fix}$ for which $N =(1/s)e^{s \tau_{fix}}$.  Solving for the former provides $\tau_{fix} \approx ln(Ns)/s$.  The successional regime occurs when $\tau_{fix} \ll \tau_{est}$, which occurs when condition
\[ N U_b \ll ln(Ns) \] 
is met for the parameter $N$, $U_b$ and $s$. 

The concurrent regime occurs when $\tau_{fix} \ge \tau_{est}$, and is characterized by the presence of multiple distinct fitness classes, which are in direct competition.  The fittest class, referred to as the nose, is the subpopulation that has not yet established and is still subject to stochastic forces.  Once a beneficial mutation establishes, a new nose subsequently forms.  The process repeats and leads to the progressive appearance of fitter classes that eventually establish and introduce variation in fitness into the population.  Selection on the other hand continuously removes this variation, which results in the population's mean fitness to increasing over time.  

Mutations and selection thus drive advancements of the population's mean fitness and nose, respectively.  These speed of the advancements are measured by the rate of increase, denoted $v$, and they are defined as the total gain in fitness divided by the associated mean time between advancements.  Both rates of increase driven by mutations and selection are determined by relative fitness gap $qs$ (discussed below).  Increases in $qs$ lead to increases in the rate of increase of both the nose in the mean fitness.  Mutation-selection balance is achieved when the two are equal, which determine specific values for the associated fitness gap $qs$ and speed of evolution $v$ at equilibrium. 

Desai and Fisher provide heuristic estimates of the approximate values for both $q$ and $v$, which are roughly equal to those obtained from a more detailed analysis.  In particular the show that,
\begin{align*}
q \sim & {2ln[Ns] \over ln[s/U_b]} \\   
\\
v \sim & {2s^2ln[Ns] \over ln^2[s/U_b]}
\end{align*}
One immediate insight that can be drawn from these expressions relates to the appearance of the $N$ and $U_b$ separately in the parameters $ln[Ns]$ and $ln[s/U_b]$, which determine the timescales of selection and mutation, respectively.  As arguments, $Ns$ directly influences the timescales of selection, while $s/U_b$ describes the respective contributions of selection and mutation to dynamics of the evolving population.  It is these contributions that set the speed of evolution $v$ in the expression above.   

\subsection*{Detailed Analysis of the Concurrent Mutations Regime}
The estimates obtained for $q$ and $v$ which results from mutation-selection balance in asexual populations following the concurrent mutation regime are confirmed with a detailed analysis of the growth dynamics at the nose and the bulk of the population.  The behavior of both depend on the stochastic variable $\tau_q$ measuring the time between the appearance of a new fittest class and its establishment.  Desai and Fisher's carefully derived estimates of the lead $qs$ and the speed of evolution $v$ at equilibrium follow from determining the expected value $\langle \tau_q \rangle$.   

To begin, it is assumed that the population is in mutation-selection balance, where in the rate of increase at the nose and that of the mean of the population is equal.  At $t=0$ the fitness class with a $(q-1)s$ selective advantage over the mean of the population, denoted $n_{q-1}(t)$, establishes by reaching size $1/qs$.  Additionally, a fitter class whose abundance is denoted $n_q(t)$ appears in the population and whose growth is initially dominated by stochastic forces.  The fitness class $n_{q-1}(t)$ contributes mutants to the size of the nose, $n_q(t)$, and each gives rise to descendants whose abundances follow identical branching process.  As a result, the moments of $n_q(t)$ can be obtained by forming the its moment generating function from the 

and eventually this fittest class establishes.
For larger and larger $t$, the growth of $n_q(t)$ becomes more and more deterministic due to selection.  At these larger time scales the abundance of this fitness class can be written as $n_q(t) = (1/qs) e^{qs(t-\tau)}$, where $\tau$ is a time dependent random variable that accounts for the stochastic contributions which over time.  In essence, $\tau$ is the offset to time for which the seemingly deterministic growth of $n_q(t)$ at large $t$ appears have reached size $1/qs$ by extrapolating backwards in time.  As $t\rightarrow \infty$, the stochastic contribution becomes less and less significant, and $\tau$ converges to a random variable denoted $\tau_q \equiv \tau (t\rightarrow \infty)$ that is independent of time.  The distribution    

Once $n_q(t)$ reaches size $1/qs$ and establishes, its growth becomes more and more deterministic due to selection.  For very large $t$, $n_q(t)$ is nearly deterministic and can be written as $n_q(t)= (1/qs)e^{qs(t-\tau)}$, where $tau$ is a time dependent stochastic variable.   

the At even larger time scales, such that $st\gg 1$ (FIX THIS), $n_q(t)$ is approximately deterministic.  However, for such large $t$, the stochasticity contributions impact the deterministic growth by altering the time for which it appears that $n_q(t)$ reached size $1/qs$ by some factor $\tau$.  Letting $\tau_q$ be the limit of $\tau$ as $t \rightarrow \infty$, then  In particular, since in the stochastic phase, which is where the primary contribution of stochasticity occurs, growth until the transition in the deterministic phase due to the influx of mutants from the subpopulation $n_{q-1}(t)$, the random variable $tau$ captures the stochasticity such that it will make the later deterministic behavior

Following a simpler birth death analysis describing the fate of a single mutant, the function $n_q(t)$ can be thought of as a composite stochastic processes, whose parts are time-dependent random variables describing the abundances of subpopulations originating from the continual appearance of lineages stemming from single mutants which arise from births in the subpopulation $n_{q-1}(t)$.  

Once this fitness class establishes, it grows exponentially as $n(t)=(1/qs)e^{(q-1)st}$ for a duration of time approximately equal to $ln(Nqs)/[(q-1)s/2]$, at which point it represents a substantial proportion of the total population.  This results in a change of mean fitness equal to  mean fitness has changed by  The fitness class class prior has a selective advantage of $qs$ over the populations mean fitness.

The . there are three parameters that determine the key behavior of adaptations in clonal populations, which are the population size $N$, the beneficial mutation rate $U_b$, and the selective advantage $s$.  What is  

The second is the strong-selection strong-mutation regime and it is referred to as the \textit{concurrent-mutations regime}. In this regime, beneficial mutations occur much more frequently, as is the case in larger populations sizes where $NU \gg 1$.  Selection is assumed to be strong and not dominated by drift.  Consequently, the time between occurrences of beneficial mutations is shorter than the time required for each to fix in the populations.  This results in the a situation where selective sweeps can overlap. This leads to what is know as clonal interference, whereby competing beneficial mutations within the population compete for fixation. Desai and Fisher point out clonal populations are known to operate in this regime, such as in some viral, bacterial and simple eukaryotic populations.  Desai and Fisher's article focuses on the clonal interference which occurs when clonal populations are in the concurrent mutations regime.  

Beneficial mutations are needed to contribute to new genetic variation, while selection eliminates on standing variation.  Desai and Fisher point out the need to understand the interplay between selection and beneficial mutations as they affect the variation in the population, in particular the selection-mutation balance.  Some comments are made on the distinct regime where mutations are weakly beneficial, in which case drift dominates and impedes their fixation through selection.  This regime occurs in small populations, where drift is much stronger than selection, or when beneficial mutations have a very weak effect, such as in the case of synonymous codon usage.  Desai and Fisher choose to focus on the moderate to large populations in strong selection regimes where drift is only important for very small subpopulations.  The figure is discussed and a comparison is drawn between the successional and concurrent mutations regimes.  As mentioned previously, the successional case occurs in small populations sizes or when $U_b$ is small so that beneficial mutations are rare and their occurrences are seperated in time.

In their model, they assume that the population size $N$ is constant.  They denote $U_b$ as the beneficial mutation rate and the fitness increase from each   mutation as $s$ (this is the log fitness).  The speed of evolution $v$ is $d\langle r \rangle / dt$ which is the rate of increase of average fitness in the population.  If the fittest mutant has advantage $ks$ over the mean, then the speed of evolution can be written as the $v=sd\langle k  / dt \rangle$.  

Desai and Fisher proceed with a simple discussion of how mutants establish in the successional regime.  Beneficial mutations occur at total rate of $NU_b$, and a mutant and the probability for it to survive drift is proportional to $s$, provided $1/N<s<1$.  Establishment occurs if the beneficial mutation survives drift, which occurs roughly when the mutant subpopulation reaches size $1/s$.  The reasoning behind this is that for a lineage of size $n$, it takes roughly $n$ generations for drift to change the population size by $n$ individuals, while selection adds $n(1+s)^n-n \approx n^2s$.  Hence if $n^2s>n$, then selection dominates drift, which exactly states that $n>1/s$ is required in order for the mutant lineage to be safe from drift. 

As a result, the mutant lineage will produce beneficial mutations that establish at a total rate of $NU_bs$ per generation, and thus, a new beneficial mutation will establish every $\tau_{establish} = 1/NU_bs$ generations.  Having reached the size $1/s$, the mutant lineage grows almost exponentially at rate $s$, which implies that the time until fixation is given by
\[
N={1 \over s}e^{s \tau_{fix}}.
\]
solving for the fixation time, we obtain $\tau_{fix}=ln(Ns)/s$ generations are required for the beneficial mutation to fix in the population.  
The calculations for the generating function are given as follows. The for

\subsection*{The Effect of Deleterious Mutations}

\subsection*{Clonal Interference and Varying "s"}

\subsection*{Conclusion}

%\section*{Appendix: Notes and Calculations}
%The focus of this paper is analyzing the dynamics of multiple mutations and their interplay in what is called clonal interference.  
%
%The successional regime occurs when beneficial mutations are rare
%
%%\begin{align*}
%%g(n,1,t)={1\over 2+s}(2+s)dt \delta_{n,0} + {1+s\over 2+s}(2+s)dt g(n,2,t-dt)+[1-(2+s)dt] g(n,1,t-dt)
%%\end{align*}

\bibliography{dfreview}
\bibliographystyle{plain}
\end{document}

