\documentclass[12pt, one column]{article}
\usepackage{amsmath}
\usepackage{amssymb}
\usepackage[margin=1.0in]{geometry}


\title{A Review of M. M. Desai's and D. S. Fisher's "Beneficial Mutation-Selection Balance and the Effect of Linkage on Positive Selection"}
\date{Fall 2016}
\author{Kevin Gomez}

\begin{document}
\maketitle
\newpage

%Writing Resources
%https://academicskills.anu.edu.au/node/492

\newpage
\section*{Introduction}
Our understanding of beneficial mutations and their dynamics is comparatively less developed than it is for neutral and deleterious mutations.  The detailed work carried out for the neutral and deleterious mutations has provided important insights into genetic diversity in populations, and they have led to extensive developments in our understanding of evolutionary processes that shape the gene pools of species.  Yet it is beneficial mutations that are the source of long-term adaptation in species.  Positive selection plays an vital role in defining the evolutionary paths taken by populations in many natural situations, but the complexity of adaptation can be formidable.  Nonetheless, progress is required in these areas since adaptation remains an important consideration for many of the key questions one may ask about the processes in evolutionary biology.  The work of Desai and Fisher provides a significant step in this direction.

The subject of their discussion is adaptation induced by beneficial mutations occurring in large and asexually reproducing populations under the conditions where selection dominates drift and there is no recombination.  Desai and Fisher focus specifically on the more interesting situation that arises when beneficial mutations occur frequently, and where multiple mutant lineages are present in the population.  In this case populations are said to be in what Desai and Fisher refer to as the concurrent-mutations regime.  The alternative situation, where beneficial mutations are rare, is referred to as the successional regime.  Adaptation in this second regime has very simple behavior, and it is characterized as nothing more than a series of successive sweeps by single mutant lineages.  The dynamics of this successional case are clear, and for this reason, the focus is drawn towards the concurrent mutations regime.  Much is less is understood about adaptation in this scenario.

There are two important effects that occur in the concurrent-mutations regime, and each influences the behavior of adaptation.  The first of these is \textit{clonal interference}, and it refers to the situation in which the fate of mutant lineages arising from a beneficial mutations, or combination of them, is altered by the presence of another superior mutant lineage.  The outcompeted lineages are driven to extinction, and the beneficial mutations that gave rise to them are lost.  The wasted beneficial mutations do not contribute to adaptation, and so the speed of evolution increases less and less as the frequency with which beneficial mutations appear in the population increases.  The second effect is the multiple mutations effect, and it refers to the accumulation of multiple beneficial mutations in lineages prior to any one of them sweeping in the population to become the dominant subpopulation.  This effect leads to the population have multiple mutant lineages present and each having a varying number of beneficial mutations.    

Desai and Fisher's analysis of the concurrent-mutations regime assumes all beneficial mutations have identical fitness effects, which has important consequences for the distribution of fitness within the population.  The only lineages that arise are those from the multiple mutations effect. Clonal interference is limited to competition between lineages with varying numbers of beneficial mutations.  The population can therefore be divided into subpopulations that are each composed of identically fit individuals which all have the same number of beneficial mutations.  These discrete groups are referred to as fitness classes by Desai and Fisher.  As will be discussed in more detail, this framework is extremely useful for developing a detailed description of adaptation in the concurrent-mutation regime, and its usefulness extends beyond the work discussed in this paper.  Moreover, the model that results from their analysis can be used to examine additional situations which include clonal interference effects that are due to beneficial mutations of varying fitness effects.  This review aims to provide an overview of Desai and Fisher's model for the concurrent-mutations regime, their examinations of the effects of deleterious mutations on adaptation, and the importance of the multiple mutations effect in populations that experience clonal interference.

\section*{The Central Ideas Behind the Model}
The heuristic analysis presented by Desai and Fisher yields a clear discussion of the dynamics of adaptation, and it provides an excellent examination of the concepts used in their model of the concurrent-mutations regime.  Among these are the separation between deterministic and stochastic behavior in growth, the time scales key events in adaptation and their relative sizes in the different regimes, and mutation-selection balance.  We review each of these in order to permit a clearer discussion of the analysis that describes their simplest model.     

The assumptions of the model imply that the growth of each fitness class will either be deterministic or stochastic depending on its size.  In the model, the parameters are the fixed population size $N$, the beneficial mutation rate $U_b$, and the common selective advantage of beneficial mutations $s$.  Since selection dominates drift, meaning $1/s \ll N$, a separation of stochastic and deterministic behavior of fitness classes.  For a fitness class of size $n$, random drift is on the order of $n$ generations.  If the selective advantage of the fitness class is $s$, then the change in the number of individuals due to selection is approximately $n^2s$.  If $n^2s \ge n$, or equivalently $n>1/s$, then selection dominates.  Otherwise, drift dominates and the fitness classes can be treated stochastically in isolation since $n \ll N$.  

In the successional regime, one ruling class generates mutants at an average rate of $N U_b$.  A beneficial mutation is said to establish when the subpopulation with the mutation grows to roughly the size of $1/s$, at which point selective forces ensure the survival of the mutation.  Only a certain percentage of the beneficial mutations that appear are able to survive stochastic forces and establish.  The probability of fixation gives the fraction that do survive and it is proportional to $s$.  In Desai and Fisher's model this probability is equal to $s$.  The waiting time between appearances of beneficial mutations that are destined to establish is exponentially distributed with mean $1/NU_b s$, and so, the expected time between establishments is $\langle \tau_{est} \rangle = 1/NU_b s$.  Once a beneficial mutation establishes, selection leads to increases in the abundance of the subpopulation carrying the mutation until its size becomes roughly $N$, at which point the mutation is said to have fixed in the population.  During that time, the growth of that subpopulation is approximately exponential with rate $s$, and so its abundance $n(t)$ is roughly equal to $(1/s) e^{st}$.  The amount of time required for a beneficial mutation to fix in the population, $\tau_{fix}$, can be solved for in terms of the parameters of $N$ and $s$ by setting $n(tau_{fix})=N$, and doing so provides $\tau_{fix} \approx ln(Ns)/s$.  It can be concluded that $\tau_{fix} \ll \tau_{est}$, or rather $N U_b \ll ln(Ns)$, in order for a population to be in the successional regime. 

The concurrent regime occurs when $\tau_{fix} \ge \tau_{est}$, and it is characterized by the presence of multiple distinct fitness classes, which are in direct competition.  The fittest class, referred to as the nose, is the subpopulation that has not yet established and is still subject to stochastic forces.  Once a beneficial mutation establishes, a new nose subsequently forms.  The process repeats and leads to the progressive appearance of fitter classes that eventually establish and introduce variation in fitness into the population.  Selection on the other hand continuously removes this variation, which results in the population's mean fitness increasing over time.  Selection thus drives advancements of the population's mean fitness, while mutations are responsible for advancements of the nose.  These rate of these advancements are measured by the rate of increase, denoted $v$, and they are defined as the total gain in fitness divided by the associated mean time between advancements.  Both rates of increase will depend on the fitness gap between the fitness of the fittest fitness class and the populations mean fitness, which may be written as the product $qs$; increases in $qs$ induce increases in each of them.  Mutation-selection balance is achieved when the two are equal, which determine specific values for the associated fitness gap $qs$ and speed of evolution $v$ at equilibrium. 

Desai and Fisher provide heuristic estimates of the approximate values for both $q$ and $v$, which are roughly equal to those obtained from a more detailed analysis.  In particular, they show that
\begin{equation}
\begin{aligned}
q \sim & {2\ln[Ns] \over \ln[s/U_b]} \\   
\\
v \sim & {2s^2\ln[Ns] \over \ln^2[s/U_b]}
\end{aligned}
\end{equation}
One immediate insight that can be drawn from these expressions relates to the appearance of the $N$ and $U_b$ separately in the parameters $\ln[Ns]$ and $\ln[s/U_b]$, which determine the timescales of selection and mutation, respectively.  As arguments, $Ns$ directly influences the timescales of selection, while $s/U_b$ describes the respective contributions of selection and mutation to dynamics of the evolving population.  It is these contributions that set the speed of evolution $v$ in the expression above.   

\section*{Detailed Analysis of the Concurrent Mutations Regime}
The estimates for $q$ and $v$, which result from mutation-selection balance in asexual populations following the concurrent mutation regime, are confirmed with a detailed analysis of the growth dynamics at the nose and the bulk of the population.  The behavior of each depends on the stochastic variable $\tau_q$ which measures the time between the appearance of a new fittest class and its establishment.  Carefully derived estimates of the lead $qs$ and the speed of evolution $v$ at equilibrium follow from determining the expected value $\langle \tau_q \rangle$.   

As previously mentioned, a population in mutation-selection balance will have a rate of increase at the nose equal to the rate of increase of the mean of the population.  At $t=0$ the fitness class with a $(q-1)s$ selective advantage over the mean of the population, whose abundance is $n_{q-1}(t)$, establishes by reaching size $1/qs$.  Concurrently, a fitter class with selective advantage $qs$ and abundance $n_q(t)$ appears in the population.  Changes in $n_q(t)$ are initially dominated by stochastic forces due to its small size, and its growth is due to the appearance of new mutants and their descendants contributed by the $(q-1)s$ fitness class. Consequently, the stochastic nature of $n_q(t)$ is characterized by growth of these mutant lineages that all follow identical branching process.  Using moment generating functions, estimates for the moments of $n_q(t)$ can be obtained as needed.

The detailed analysis leading to a careful description of $n_q(t)$ provides a way of measuring the time until establishment for the nose.  For large $t$, the growth of $n_q(t)$ becomes more and more deterministic due to selection.  At these larger time scales the abundance of this fitness class can be written as $n_q(t) = (1/qs) e^{qs(t-\tau)}$, where $\tau$ is a time dependent random variable that accounts for the stochastic contributions which accumulate over time.  In essence, $\tau$ is the offset to time for which the seemingly deterministic growth of $n_q(t)$ at large $t$ appears have reached size $1/qs$ by extrapolating backwards in time.  As $t\rightarrow \infty$, the stochastic contribution becomes less and less significant, and $\tau$ converges to a random variable denoted $\tau_q \equiv \tau (t\rightarrow \infty)$ that is independent of time.  Since $\tau = t +(1/qs)(\ln(1/qs)-\ln(n_q(t)))$, they obtain the moments of $\tau$ from those of $n_q(t)$.  By taking the limit of $\tau$ as $t\rightarrow \infty$, they obtain an expression for $\langle \tau_q \rangle$ below, as well as higher moments. 
\begin{equation} 
\langle \tau_q \rangle = \frac{1}{(q-1)s}\ln\left(\frac{s}{U_b}\frac{(q-1)sin(\pi/q)}{\pi^{\gamma/q}}\right)
\end{equation}
Steady state occurs when the rates of increase of the population's mean fitness and the nose are equal, and this is achieved for a specific value of $q$.  Equation (2) can be into $n_q(t)$ to track the growth over time of a fitness class beginning at the nose, which then allows us to solve for the steady state $q$.  The mean fitness increases by $s$ in the average establishment time $\langle \tau_q \rangle$.  Once the nose establishes, the corresponding fitness class grows at the rate of $(q-1)s$ until the next establishment, assuming that $q>2$ to ensure that growth does not slow due to saturation.  At that point the mean has increase again by $s$, and the rate of growth is diminished to $(q-2)s$ for another $\langle \tau_q \rangle$ until the next establishment.  This continues until this same class becomes the largest fitness class of size approximately equal to $N$, which occurs in a time span of $(q-1) \langle \tau_q \rangle$.
\begin{equation}
N \approx  \frac{1}{qs} \exp\left[ \frac{q(q-1)s \langle \tau_q \rangle}{2}\right]  
\end{equation}
This is a transcendental equation in $q$, which can be solved using iterations.  A simple zeroth order approximation of $q$ can be used to solve for its dependence on the parameters, and to estimate the rate of evolution $v$. The expressions that follow are similar to those obtained in the heuristic analysis.
\begin{equation}
\begin{aligned}
q \approx & {2\ln[Ns] \over \ln[s/U_b]} \\   
\\
v \approx & s^2 \left[{2\ln[Ns] - \ln[s/U_b] \over \ln^2[s/U_b]}\right]
\end{aligned}
\end{equation} 
The dependence of steady state rate of evolution $v$ on the with of the fitness distribution of the bulk obeys Fisher's fundamental theorem of natural selection.  The growth or decline of fitness classes over a time period of $\tau_q$ is exponential with a rate dependent that depends on how many more (or less) mutations it has above (or below) the mean.  If a particular fitness class has $l$ mutations  more (or less) than the mean, its size will be approximately $N \exp[-\sum_{i=1}^{l} \tau_q] \sim N \exp[(\tau_q/2s)(ls)^2]$.  The variance of this Gaussian fitness distribution, is $\sigma^2 = (s/\tau_q)=v$ which is the rate of evolution.  Furthermore, since the above indicates that $N \exp[(qs)^2/2\sigma^2]\approx (1/qs)$, it follows that $v=\sigma^2 \approx (qs)^2/2\ln(Ns)$.  The magnitude of the lead is the key measure of the fitness distribution's width, and the equality between $v$ and $\sigma^2$ follows specifically from their relationship to the lead.

\section*{The Effect of Deleterious Mutations}
The simple model with a single beneficial $s$ does not incorporate the occurrence of deleterious mutations, which can impact the dependence of $v$ on the mutation rate and population size.  In their analysis, deleterious mutations of various fitness effects are considered in the framework of the simple model.  The magnitudes of these deleterious mutations are denoted $s_d$.  The authors consider two ranges of fitness effects separately where either $s_d \geq s$ or $s_d \ll s$.  They show that in both cases deleterious mutations decrease the mean fitness of the population and reduce the rate at which new mutants establish.  Since the first of these increases the speed of evolution and the second decreases it, the two effects either completely or partially cancel one another depending on the case considered.  They conclude that deleterious mutations do substantially change the rate of evolution predicted by the simple model.     

In the first case where deleterious fitness effects satisfy $s_d \geq s$, Desai and Fisher argue that the rate of evolution remains the same.  The mutation rate for deleterious mutations in this range is denoted $U_d^{>}$.  These mutations lower the mean fitness of the population by $U_d^{>}$ through their effects on the largest subpopulation.  Since $s_d \geq s$, mutation-selection balance for deleterious mutations is achieved on a time scale $1/(s_d +U_d^{>}) < 1/s$.  The equilibrium frequency of deleterious mutants among the largest subpopulation is $U_d^{>}/s_d$ and their relative fitness is $-s_d$.  The portion of the large subpopulation has frequency $(1-U_d^{>}/s_d)$ with not deleterious mutations will have relative fitness $0$.  The change in the mean fitness of the populations is $-s_d (U_d^{>}/s_d) = -U_d^{>}$.  The increase in the lead of the nose by $U_d^{>}$ produces an increase in its growth rate by that amount.  On the other hand, deleterious mutations also occur at the nose at a rate of $U_d^{>}$, which decreases the rate of growth of mutant lineages the nose by that same amount.  The two effects cancel out as a result, and the speed of evolution $v$ remains the same.

In the case of weakly deleterious mutations ($s_d \ll s$) there will be small changes in the speed of evolution $v$.  The mutation rate for weakly deleterious mutations is denoted $U_d^{<}$.  Beginning with the effects at the nose, weakly deleterious mutations will decrease the size of nose advancement from $s$ to $s$ minus an amount that is determined by the mutational load created by deleterious mutations.  The size of the decrease will depend on the size of $U_d^{<}$ relative to $s$.  When $(U_d^{<} / qs)\ln(s/U_b) <<1$ (or roughly $U_d^{<}/s \ll 1$), the rate at which deleterious mutations occur is smaller than the mean rate of establishments occurring at the nose.  The load placed on nose extending mutations reduces the effective $s$ by approximately $(U_d^{<} s_d /(q-1)s)\ln(s/Ub)$, which is much smaller than $s$ given the assumption above.  When $U_d^{<} / s >> 1 $, the rate at which deleterious mutations occur is larger than the mean rate of establishments.  The load placed on nose extending mutations is larger as a result, but it is still on the order of $s_d$ whose size is still significantly smaller than $s$.  In both cases, whether $U_d^{<}/s \ll 1$ or $U_d^{<}/s >> 1$, the effects of deleterious mutations at the nose decreases the rate of growth of mutant lineages by an amount on the order of $s_d$ or less.  In the bulk of the population, deleterious mutations once again lead to a reduction in the mean fitness.  However, since $s_d \ll s$, mutation-selection balance for deleterious mutations in the bulk is achieved on time scales larger than those corresponding to the mean time between establishments $\tau_q$.  The total effect on mean fitness is cumulative over the period of time in which a fitness class at the nose sweeps in the population, i.e. the fitness class becomes the dominant class of the bulk.  This duration of time is roughly equal to $qs/v \approx \ln(s/U_b)/s$.  The reduction to the mean fitness remains similar to the prior description, amounting to a reduction of at most order $U_d^{<}$ and maximized when $1/s_d$ is on the order of the sweep time.  Unfortunately, the precise effect on $v$ from the combined contributions mentioned above cannot be given using this simple analysis.  This may seem like an unsatisfactory conclusion, but additional work has been carried out for this particular scenario in subsequent papers by Good and Desai (2014).

\section*{Clonal Interference and the Multiple Mutations effect with Varying ``s"}
As was mentioned in the introduction, the assumption that all beneficial mutations have the same effect size $s$ eliminates the need to consider clonal interference due to competing beneficial mutations with different fitness effects occurring on the same genetic background.  Generally, beneficial mutations will have different fitness effects, and in fact, evolutionary experiments have been used to estimate what the distribution of fitness effects are.  Desai and Fisher devote a section to examining adaptation in the concurrent-mutations regime with varying $s$.  Their discussion does not attempt to provide a rigorous analysis, but rather, it aims to use results of the model to examine what the qualitative effects of varying $s$ are on adaptation.  In particular, the authors are able to point out that in a regime where clonal interference occurs, multiple mutations must be accounted for.  Below, I'll present a subset of the results derived from the standard clonal interference analysis, and proceed with discussing a specific case examined by Desai and Fisher that uses the results of the simple model to properly account for the effects of both multiple mutations and clonal interference on adaptation in the concurrent-mutation regime.

In the recent work by Gerrish and Lenski (1998) on clonal interference, as well as subsequent work by others, the effect of multiple mutations is discounted .  Their analyses yield an characteristic fitness effect $s_{CI}$, which estimates the average  that depends on the population size $N$, which will determine the rate at which adaptation progresses $v_{CI} \sim [s_{CI}]^2$.  The relationship between $s_{CI}$ and population size $N$ will depend on the distribution of selective advantages $\rho(s)$, and it is approximately given by $-\\ln[U_b p(s_{CI})] \approx \\ln[N s_{CI}]$.  The analysis provides incorrect predictions about the relationship between the beneficial mutation rate $U_b$ and characteristic size $s_{CI}$ of mutations that fix, and this is due to the discounting of multiple mutations.  As Desai and Fisher demonstrate in their paper, multiple mutations are relevant in populations that are large enough to experience clonal interference.  Clonal interference impeding the fixation of a beneficial mutation of size $s$ will occur when the establishment rate of mutations superior to it, which is greater then or equal to $NU_b^{>s}s$, exceeds the fixation time of a mutations of size $s$, $s/\\ln(Ns)$.  The results of the single $s$ model show that multiple mutations occur since $NU_b^{>s} \gg 1/\\ln(Ns)$.  Both clonal interference and multiple mutations play a role in determining the behavior of adaptation in the concurrent mutations regime with varying $s$.

To consider the contribution of multiple mutations, we must account for how clonal interference and multiple mutations interact.  In such a populations, a predominant set of fitness effects emerges whose sizes will all be roughly in the range of some $\tilde{s}$ occurring with a mutation rate of $\tilde{U}_b$.  To see why this is so, Desai and Fisher start by consider an initially clonal population, where mutations of varying sizes appear in either the initial genetic background, or in prior mutant lineages that have already arisen from the initial population. Double mutants, triple mutants and so on will appear through the effects of multiple mutations.  Eventually, a fittest mutant will give rise to a lineage that establishes before any others manage to fix in the population, driving them to extinction.  The typical fitness effect sizes that produce such predominant mutations, will depend on the population size, the beneficial mutation rate, and the distribution of selective advantages.  The value of $\tilde{s}$ cannot be determined as readily as $s_{CI}$.  However, the existence of such a range near some value $\tilde{s}$ follows from the fact that mutations much smaller than $\tilde{s}$ are routinely outcompeted, while those significantly greater than $\tilde{s}$ are much too rare by definition to prevent the fixation of mutations in the predominant range.  Both $\tilde{s}$, and more importantly the range of the range of around it, and $\tilde{U}_b$, will depend on the the parameters $N$, $U_b$, and the distribution of selective advantages $\rho(s)$.  Unlike the characteristic selective advantage $s_{CI}$ obtained in the clonal interference analysis, the rate of increase in $\tilde{s}$ due to increases in $N$ and $U_b$ will be less.

Given a range of predominant mutations that is sufficiently narrow, the speed of evolution $v(s)$ from the single $s$ model can be used to determine the predominant $s$, $\tilde{s}$, along with an approximation for $v$.  This can be achieved by using the mutation rate $U_b$ typical to the predominant range, and calculating $\tilde{s}$ with corresponding $v$ from equation (5).
\begin{equation}
v = \max_{s} v(s)=v(\tilde{s})
\end{equation}

The result can be checked for consistency, in the particular case that we review here which correspond to a case where the distribution of fitness effects has a short tail.  This is done by prescribing a functional form for the mutation rate of beneficial mutations of size $s$ given by $\mu(s)=U_b \rho(s)$, and examining the two cases for the shape of $\mu(s)$.  Here $\mu(s)$ is taken as the function $\mu(s) = \exp[-\ell - (s/\sigma)^\beta]$, where $U_b \propto \exp[-\ell]$ so that $\ell$ provides a parameter for mutations, $\sigma$ is the characteristic selective advantage, and $\beta $ prescribes the size of the distributions tail.  The additional parameters $L=\\ln(N\sigma)$ and $\Lambda(s)= \ln(1/\mu(s))$ can be defined.  In the case where $\beta > 1$, the tail of $\rho(s)$ is short.  When the population size is sufficiently large and $2L/\Lambda(s)\gg 1$, the predominant $s$ approximation yields equation (6) for $\tilde{s}$ and $v$ using the large $q$ result from the single $s$ model.
\begin{equation}
\begin{aligned}
\tilde{s} &= \sigma\left[ \frac{\ell}{\beta-1} \right]^{1/\beta}\\
\\
v & \approx C_\beta \sigma^2 \frac{2 \ln(N\sigma)}{\ell^{2-2\beta}}
\end{aligned}
\end{equation}
Solving for the value of $q$ provides that it is indeed in the range used to obtain the estimate, mainly that $q=2L(\beta-1)/\ell \beta$ is fairly large as expected.  The expression for $v$ in (6) shows that the velocity estimations from the single $s$ model provide a consistent estimate of the speed of evolution for this particular case by properly accounting for the dependence of $\tilde{s}$ and $\tilde{U}_b$ on $N$ and $U_b$.  Other scenarios are not as easily analyzed, and some of the assumptions necessary to apply the results from the simple model breakdown. However, Desai and Fisher's arguments do provide the correct direction that must be taken for further analysis of the concurrent-mutations regime with clonal interference.

\section*{Conclusion}
%Desai and Fisher's analysis of adaptation in the concurrent-mutations regime for asexually reproducing populations with likage provides a significant contribution to our understanding of adaptation.  Their analyis leads to a proper quanitification of the waste created to large populations that evolve in the clonal regime, and unlike the simple behavior that emerges in the successional regime, we fined that the speed of evolution changes in its relationship to the parameters for population size, mutations, and selection.
%
% beneficial mutations through clonal interference, where in these small effect beneficial mutations are outcompeted by large effect ones.  The application of Garrish and Lenski's (1998) analysis which derives a minimal effect size sci which fixes in the wild type, and leads to a successional mutation behavior.  The dependence of sci on the population size N following Garrish and Lenski's analysis provides $\ln[1/U_b p(s_{CI})]\approx \ln[Ns_{CI}]$ ($p(s)$ is distribution of fitness effects “$s$”), which makes a crucial and incorrect assumption in which double mutants do not occur, and results in incorrect predictions. Populations large enough for clonal interference to matter are also large enough for bound mutants to repeatedly appear.  Clonal interference can affect the fixation of a mutation of size s only when the establishment rate of  mutations stronger than $s$, which is atleast $NU_b^{>s} \gg 1/\ln(Ns)$ indicating that clonal interference will play a role, then so too will mutltiple mutations (above is roughly the condition for concurrent mutations = multiple mutations). 

\bibliography{dfreview}
\bibliographystyle{plain}
\end{document}
