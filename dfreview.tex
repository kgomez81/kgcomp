\documentclass[12pt]{article}
\usepackage{amsmath}
\usepackage{amssymb}
\usepackage[margin=1.0in]{geometry}


\title{A Review of M. M. Desai's and D. S. Fisher's "Beneficial Mutation-Selection Balance and the Effect of Linkage on Positive Selection"}
\date{Fall 2016}
\author{Kevin Gomez}

\begin{document}
\maketitle
\newpage

%Writing Resources
%https://academicskills.anu.edu.au/node/492

\newpage
\subsection*{Introduction}
Our understanding of beneficial mutations and their dynamics is comparatively less developed than it is for neutral and deleterious mutations.  The detailed work carried out for the neutral and deleterious mutations has provided important insights into genetic diversity in populations, and they have led to extensive developments in our understanding of evolutionary processes that shape the gene pools of species.  Yet it is beneficial mutations that are the source of long-term adaptation in species.  Positive selection plays an vital role in defining the evolutionary paths taken by populations in many natural situations, but the complexity of adaptation can be formidable.  Nonetheless, progress is required in these areas since adaptation remains an important consideration for many of the key questions one may ask about the processes and observations in evolutionary biology.  The work of Desai and Fisher provides a significant step in this direction.

In their paper, Desai and Fisher focus on adaptation in large asexually reproducing populations with no recombination, in which beneficial mutations are common and selection dominates drift.  In this setting, multiple beneficial mutations are present within the population, while linkage ties them to lineages in which they arise.  When populations evolve in this manner, superior mutant lineages will out-compete other , and this results in wasted proportion of beneficial mutations that will depend on the interactions of selection and mutation,  which makes this work so important.  Examples found in nature including viruses, bacteria, and simple eukaryotes, which evolve in conditions quite similar to this.   

 brings about two important effects.  The first of these is called \textit{clonal interference}, which occurs when distinct beneficial mutations that arise in the same lineage compete with each other.  Eventually, the frequency of the subpopulation possessing the beneficial mutation with the largest fitness effect, reaches the value one.  Such beneficial mutations is said to have fixed in the population when this occurs.  The second affect refers to multiple mutations arising in the same lineage, such as double mutants.

Desai and FisterThe analysis Such a population can be decomposed into subpopulations consisting of identically fit individuals, which are called fitness classes.  This grouping, which may pool individuals with distinct beneficial mutations, allows for the . 

 

These mutations will be present and in direct competition with each other if not linked critical to extend our understanding of adaptation, since it more often than not, it is a key evolutionary process actively molding the fate of 
and where one ignores occurrences of neutral and deleterious mutations.  Scenarios of this type have been examined in the context of examining questions about the ubiquity of sex

the strength of selection and the rate of mutations  

Interest in these scenarios dates   

The work of Desai and Fisher provides a careful analysis for such important contribution to the existing body of theoretical literature on beneficial mutations, and in particular, their detailed analysis provides a solid framework for describing adaptation in asexually reproducing populations with no recombination.

The rate at which beneficial mutations occur is a key contributor to the dynamics of long-term adaptation, and their selective advantage plays a role in .  There are two regimes of interest, which have been investigated to some extent in past work.  

%explain fitness classes

%In the first of these, positive selection is characterized by fairly simple dynamics, which are well understood.  The second regime leads to much more complicated behavior, where prior efforts to properly analyze its dynamics have been incomplete in their descriptions.        

%Joanna mentioned that I should cut the explanation of the two regime crap and focus on the explaining why i chose to focus on the concurrent mutations regime, and why its important in reference to the past work of evolution with beneficial mutations focusing on sexuals without linkage (good place to define the notion of fitness class).  More importantly, as a general point, in each section, give a reason why you chose to discuss the information you included.  Why is it important?  In reference to your interest and future work, past work and the mathematics of prior population genetics.

%make this a simple sentence (who cares about this shit!!! 
The simplest regime, known as the strong-selection weak-mutation regime, occurs when the average time between appearances of beneficial mutations is much larger than the time required for selection to fix a beneficial mutation.  In this particular scenario, selection purges variation in fitness much faster than it can be introduced into the population, which results in a succession of a single ruling classes composed of identically fit individuals.  For this reason, it is referred to as the successional regime.  

The second is the strong-selection strong-mutation regime, which occurs when the time between appearances of beneficial mutations is comparable to or smaller than time needed for beneficial mutations to fix in the population.  In this particular regime, selection fails to purge all variation in fitness before more is introduced with the appearance of the next beneficial mutation.  It is also denoted the concurrent mutations regime due to the fact that these dynamics give rise a population with **concurrent clonal subpopulations, referred to as fitness classes, since subpopulations posses distinct fitness levels, and each is composed of individuals with identical fitnesses** (explain fitness classes).  This second regime is the focus of the paper, starting with the simplest model, followed by an exploration of its application to more complicated situations associated with the relaxation of certain assumptions.  Desai and Fisher's complete work is extensive and far reaching in its results, but I will limit myself to covering their simplest model, with only a subset of its extensions. 

\subsection*{The Central Idea Behind the Model}
In order to clarify the intuition behind the model, Desai and Fisher provide a heuristic analysis of the dynamics of both selection and mutations, which lead to powerful insights and results that aid in formulating the concept of their model.  Their basic approach starts with the same set of assumptions that are later adopted in the more careful approach taken.  These assumptions consist of considering asexually reproducing populations with no recombination and fixed population size $N$, which give rise to beneficial mutations at rate $U_b$.  The selective advantage of each beneficial mutation is a constant $s$.  Its magnitude is assumed to be large enough so that $1/s \ll N$, ensuring that selection dominates drift and providing a separation of stochastic and deterministic behavior attributable to fitness classes based on their size.  A fitness class of size $n$, it takes random drift at least $n$ generations to decrease its size by $n$.  If the selective advantage of the subpopulation is $s$, then in $n$ generations its size will increase on average by approximately $n(1+s)^n-n \approx n^2s$.  Consequently, if $n^2s \ge n$, or equivalently $n>1/s$, then selection dominates.  Otherwise, drift dominates since $n \leq 1/s \ll N$, and so these subpopulations must be treated stochastically.

When mutations are rare and selection is strong, one ruling class generates mutants at an average rate of $N U_b$.  A beneficial mutation is said to establish when the subpopulation with the mutation grows to roughly the size of $1/s$, at which point selective forces ensure the survival of the mutation.  Only a certain percentage of the beneficial mutations that appear are able to survive stochastic forces and establish.  The probability of fixation gives the fraction of them that do survive and it is proportional to $s$.  In Desai and Fisher's simple model this probability is equal to $s$.  The waiting time between appearances of beneficial mutations that are destined to establish is exponentially distributed with mean $1/NU_b s$.  

Once a beneficial mutation establishes, selection increases the abundance of the subpopulation carrying the mutation until its size is roughly $N$, at which point the beneficial mutation is said to have fixed.  During the time the growth of the subpopulation with the beneficial mutation is exponential with rate $s$, and so its   size $n(t)$ is roughly equal to  $(1/s) e^{st}$.  The amount of time required for fixation to occur,  $\tau_{fix}$, can be solved for in terms of the parameters of $N$ and $s$.  Doing so provides $\tau_{fix} \approx ln(Ns)/s$.  The successional regime occurs when the time until fixation is much smaller than the mean time between establishments, or rather $\tau_{fix} \ll \tau_{est}$.  This occurs when below is met.
\[ N U_b \ll ln(Ns) \] 

The concurrent regime occurs when $\tau_{fix} \ge \tau_{est}$, and is characterized by the presence of multiple distinct fitness classes, which are in direct competition.  The fittest class, referred to as the nose, is the subpopulation that has not yet established and is still subject to stochastic forces.  Once a beneficial mutation establishes, a new nose subsequently forms.  The process repeats and leads to the progressive appearance of fitter classes that eventually establish and introduce variation in fitness into the population.  Selection on the other hand continuously removes this variation, which results in the population's mean fitness  increasing over time.  

Selection and mutations thus drive advancements of the population's mean fitness and nose, respectively.  These speed of the advancements are measured by the rate of increase, denoted $v$, and they are defined as the total gain in fitness divided by the associated mean time between advancements.  Both rates of increase driven by mutations and selection are determined by relative fitness gap $qs$ (discussed below).  Increases in $qs$ lead to increases in the rate of increase of both the nose in the mean fitness.  Mutation-selection balance is achieved when the two are equal, which determine specific values for the associated fitness gap $qs$ and speed of evolution $v$ at equilibrium. 

Desai and Fisher provide heuristic estimates of the approximate values for both $q$ and $v$, which are roughly equal to those obtained from a more detailed analysis.  In particular they show that
\begin{equation}
\begin{aligned}
q \sim & {2ln[Ns] \over ln[s/U_b]} \\   
\\
v \sim & {2s^2ln[Ns] \over ln^2[s/U_b]}
\end{aligned}
\end{equation}
One immediate insight that can be drawn from these expressions relates to the appearance of the $N$ and $U_b$ separately in the parameters $ln[Ns]$ and $ln[s/U_b]$, which determine the timescales of selection and mutation, respectively.  As arguments, $Ns$ directly influences the timescales of selection, while $s/U_b$ describes the respective contributions of selection and mutation to dynamics of the evolving population.  It is these contributions that set the speed of evolution $v$ in the expression above.   

\subsection*{Detailed Analysis of the Concurrent Mutations Regime}
These estimates for $q$ and $v$, which result from mutation-selection balance in asexual populations following the concurrent mutation regime, are confirmed with a detailed analysis of the growth dynamics at the nose and the bulk of the population.  The behavior of both depends on the stochastic variable $\tau_q$ measuring the time between the appearance of a new fittest class and its establishment.  Desai and Fisher's carefully derived estimates of the lead $qs$ and the speed of evolution $v$ at equilibrium follow from determining the expected value $\langle \tau_q \rangle$.   

As previously mentioned, a population in mutation-selection balance will have a rate of increase at the nose equal to the rate of increase of the mean of the population.  At $t=0$ the fitness class with a $(q-1)s$ selective advantage over the mean of the population, whose abundance is $n_{q-1}(t)$, establishes by reaching size $1/qs$.  Concurrently, a fitter class with selective advantage $qs$ and abundance $n_q(t)$ appears in the population.  Changes in $n_q(t)$ are initially dominated by stochastic forces due to its small size, and its growth is due to the appearance of new mutants and their descendants contributed by the $(q-1)s$ fitness class. Consequently, the stochastic nature of $n_q(t)$ is characterized by growth of these mutant lineages that all follow identical branching process.  Using moment generating functions, estimates for the moments of $n_q(t)$ can be obtained as needed.

The detailed analysis leading to a careful description of $n_q(t)$ provides a way of measuring the time until establishment for the nose.  For large $t$, the growth of $n_q(t)$ becomes more and more deterministic due to selection.  At these larger time scales the abundance of this fitness class can be written as $n_q(t) = (1/qs) e^{qs(t-\tau)}$, where $\tau$ is a time dependent random variable that accounts for the stochastic contributions which accumulate over time.  In essence, $\tau$ is the offset to time for which the seemingly deterministic growth of $n_q(t)$ at large $t$ appears have reached size $1/qs$ by extrapolating backwards in time.  As $t\rightarrow \infty$, the stochastic contribution becomes less and less significant, and $\tau$ converges to a random variable denoted $\tau_q \equiv \tau (t\rightarrow \infty)$ that is independent of time.  Since $\tau = t +(1/qs)(ln(1/qs)-ln(n_q(t)))$, we can obtain the moments of $\tau$ from those of $n_q(t)$.  By taking the limit of $\tau$ as $t\rightarrow \infty$, we obtain an expression for $\langle \tau_q \rangle$ below, as well as higher moments. 
\begin{equation} 
\langle \tau_q \rangle = \frac{1}{(q-1)s}ln\left(\frac{s}{U_b}\frac{(q-1)sin(\pi/q)}{\pi^{\gamma/q}}\right)
\end{equation}
Steady state occurs when the rates of increase of the population's mean fitness and the nose are equal, and this is achieved for a specific value of $q$.  Equation (2) can be used to track the growth over time of a fitness class beginning at the nose, which then allows us to solve for the steady state $q$.  The mean fitness increases by $s$ in the average establishment time $\langle \tau_q \rangle$.  Once the nose establishes, the corresponding fitness class grows at the rate of $(q-1)s$ until the next establishment, assuming that $q>2$ to ensure that growth does not slow due to saturation.  At that point the mean has increase again by $s$, and the rate of growth is diminished to $(q-2)s$ for another $\langle \tau_q \rangle$ until the next establishment.  This continues until this same class becomes the largest fitness class of size approximately equal to $N$, which occurs in a time span of $(q-1) \langle \tau_q \rangle$.
\begin{equation}
N \approx  \frac{1}{qs} \exp\left[ \frac{q(q-1)s \langle \tau_q \rangle}{2}\right]  
\end{equation}
In this expression we may substitute in for $\langle \tau_q \rangle$ to obtain an transcendental equation in $q$, which can be solved using iterations.  A simple zeroth order approximation of $q$ can be used to solve for its dependence on the parameters, and an estimate of rate of evolution $v$. The expressions that follow are identical to those obtained in the heuristic analysis.
\begin{equation}
\begin{aligned}
q \approx & {2ln[Ns] \over ln[s/U_b]} \\   
\\
v \approx & s^2 \left[{2ln[Ns] - ln[s/U_b] \over ln^2[s/U_b]}\right]
\end{aligned}
\end{equation} 
The resulting steady state rate of evolution $v$ bears a relationship to the fitness distribution of the bulk, that follows the Fisher's fundamental theorem of natural selection.  The growth and decline of a fitness classes over a time period of $\tau_q$, follow the description above, and hence its growth (or decline) is exponential with a rate dependent that depends on how many more (or less) mutations it has above (or below) the mean.  If a particular fitness class has $l$ mutations  more (or less) than the mean, its size will be approximately $N \exp[-\sum_{i=1}^{l} \tau_q] \sim N \exp[(\tau_q/2s)(ls)^2]$.  The variance of this Gaussian fitness distribution, is $\sigma^2 = (s/\tau_q)=v$ which is the rate of evolution.  Furthermore, since the above indicates that $N \exp[(qs)^2/2\sigma^2]\approx (1/qs)$, it follows that $v=\sigma^2 \approx (qs)^2/2ln(Ns)$.  The magnitude of the lead is the key measure of the fitness distribution's width, and the equality between $v$ and $\sigma^2$ follows specifically from their relationship to the lead.

\subsection*{The Effect of Deleterious Mutations}
The simple model with a single $s$ does not incorporate the occurrence of deleterious mutations, which can impact the dependence of $v$ on the mutation rate and population size.  In their analysis, deleterious mutations with effects $s_d$ that vary in size are considered in the framework of the simple model with a single $s$ model.  Two ranges of fitness effects are examined, but in both cases, the contributions to the dynamics of the bulk and the nose act in opposition with respect to the changes they induce on their rates of increase.   

In the first range of deleterious effects corresponding to $s_d \geq s$, with mutation rate $U_d^{>}$, a qualitative analysis reveals the fitness distribution of the bulk changes, but the steady state $v$ remains unaffected.  Deleterious mutations lower the mean fitness of the population by $U_d^{>}$ through its effects on the largest subpopulation.  Since $s_d \geq s$, mutation-selection balance for deleterious mutations is achieved on a time scale $1/(s_d +U_d^{>}) < 1/s$.  At this equilibrium, the frequency of deleterious mutants is $U_d^{>}/s_d$ with relative fitness $0$ and the unaffected portion is $(1-U_d^{>}/s_d)$ with relative fitness $-s_d$, and hence the new mean is $-s_d (U_d^{>}/s_d) = -U_d^{>}$.  This increases the lead of the nose by that amount, and consequently its growth by the same amount.  At the nose, deleterious mutations occurring at rate $U_d^{>}$ simply decrease the rate of growth of mutant lineages the nose by that amount.  The two effect cancel out as a result, and the speed of evolution $v$ remains mostly unchanged.

Weakly deleterious mutations, whose mutation rate is denoted rate $U_d^{<}$, and corresponds to the case where $s_d \ll s$, affect the bulk and nose  in ways that require more careful examination.  Deleterious mutations decrease the gain in fitness to $s$ minus an amount corresponding to the mutational load which depends on the magnitude of $U_d^{<}$ relative to $s_d$.  When $(U_d^{<} / qs)ln(s/U_b) <<1$ (or equivalently $U_d^{<}/s \ll 1$), the rate of deleterious mutations is smaller than the the rate of establishments occurring at the nose.  The associated load placed on nose extending mutations is small, and is approximately $(U_d^{<} sd /(q-1)s)ln(s/Ub)$.  In this case, the effective $s$ is reduced on average by that amount, and it is significantly smaller than $s$.  When $U_d^{<} / s >> 1 $, deleterious mutations occur at a faster rate than establishments on average.  The load placed on nose extending mutations is larger as a result, but it is still on the order of $s_d$.  In both cases, the  effects of deleterious mutations on the bulk lead a reduction in the mean fitness.  However, since $s> s_d$, mutation-selection balance is achieved on time scales larger than those corresponding to the mean establishment time.  As a result, the total effect on mean fitness is cumulative over period of the sweep for which a fitness class at the nose becomes the dominant class of the bulk, roughly $qs/v \approx ln(s/U_b)/s$.  The reduction to the mean fitness remains similar to the prior description, amounting to a reduction of at most order $U_d^{<}$ and maximized when $1/s_d$ is on the order of the sweep time.  Unfortunately, the precise effect on $v$ from the combined contributions mentioned above cannot be given with this using this simple analysis.  This may seem like an unsatisfactory analysis, but additional work has been carried out for this particular scenario.  This is examined in further work by Good and Desai \cite{GoodDesai12}, and the results indicate that the first analysis was correct, while for the second case where $s_d \ll s$, the more detailed analysis confirms an identical result.

\subsection*{Clonal Interference and Varying "s"}
In Desai and Fisher's simple model, beneficial mutations all have the same effect size $s$.  This generally does not hold in most situations, which makes it necessary to consider the what the effects on adaptation might be when relaxing this assumption.  Clonal interference becomes a relevant concern in this setting.

However, its relaxation .  It which but as mentioned, the effects of beneficial mutations range in their selective advantage, and in fact, an important body of work exists that analyzes the distribution of $s$ in the natural world.  Yet, the single $s$ model can be used to examine qualitatively the some of the evolutionary aspects of concurrent mutations regime (FIX THIS) with relaxing the assumption that the fitness effects vary.  This can be used to examine evolution in the concurrent mutation regime for a setting where beneficial mutations have varying effect sizes.

The occurrences of small effect mutations leads to additional wasted beneficial mutations through clonal interference, where in these small effect beneficial mutations are out-competed by large effect ones.  The application of Garrish and Lenski's (1998) analysis which derives a minimal effect size sci which fixes in the wild type, and leads to a successional mutation behavior.

The dependence of sci on the population size N following Garrish and Lenski's analysis provides $ln[1/U_b p(s_{CI})]\approx ln[Ns_{CI}]$ ($p(s)$ is distribution of fitness effects “$s$”), which makes a crucial and incorrect assumption in which double mutants do not occur, and results in incorrect predictions. Populations large enough for clonal interference to matter are also large enough for bound mutants to repeatedly appear.  Clonal interference can affect the fixation of a mutation of size s only when the establishment rate of  mutations stronger than $s$, which is atleast $NU_b^{>s} \gg 1/ln(Ns)$ indicating that clonal interference will play a role, then so too will mutltiple mutations (above is roughly the condition for concurrent mutations = multiple mutations). 

Clonal interference and multiple mutations interact to produce a predominant range of fitness effects of roughly magnitude with mutation rate , and they accumulate in process similar to the single “$s$” model.
In an initially clonal population, there will be a fittest mutant that is established in wild type population prior to the fixation of any other mutation of combination of mutations, which will drive all other prior mutations to extinction.  These are call predominant mutations in the range of some with mutation rate.

The clonal-like-interference balanced with multiple mutation produce the predominant “$s$” due to the fact that for smaller fitness increments s are outcompeted by predominant mutants, and do not affect long term evolution, while larger “$s$” are too rare to occur in the time that one or predominant mutations arise.  Both  with mutation rate  depend on $N$ and $U_b$ as in the clonal interference analysis, but also must depend on the effects of multiple mutations.  This dependence is determined by p(s) and lead to two possibilities, excluding when the decay in $p(s) \sim 1/s^3$  or less.

Specifically, a class of distributions where is a characteristic selection coefficient, $\ell=ln(s/U_b)$ controls the mutation rate and determines the shape of the distribution of rare large mutations. Once, a the predominant range is determined, the predominant-$s$ approximation of $v(s)$ can be obtained when a narrow range of around exists, by setting $\tilde{v}* = \max_{s} v(s)$, where $v(s)$ is obtained from the velocity estimations from the single $s$ model.  Short-tailed for which leads to an dependence on $N$ and $U_b$ which is determined by  the size of , where , but can provide estimates.
When , we should obtain a “$q$” that allows for the application of (41), which provides an estimate  from maximizing $v(s)$ providing (58), and gives s that is roughly independent of $N$ and $U_b$, where .  
We also get that does not depend strongly on N, but it does increase as $U_b$ increases, following from the fact that $U_b$ leads to more multiple mutations become more important than single large mutations, which requires to grow.
This leads to identical dependence on $N$ but weaker dependence on $U_b$ in $v(s)$.
Not so large $L=ln(Ns)$, will lead to more complicated behavior due to some s crossing over into regime where so large $q$ results becomes inconsistent.

Long tailed , for which 
In the large limit $2L/\ell$ the estimates for , , and , indicates that $v(s)$ grows faster than linearly in $ln(N)$, but the dependence on $U_b$ is negligible, but does determine how large $N$ must be, while $q$ is never really large for long-tailed because  increases substantially with $N$.  In short tailed case, multiple mutations matter and contribute.
When, $q<2$ (but sticks to 2), and leads to successional behavior for which which agrees with estimate of $v(s)$ obtained for, and for this range, the clonal-interference estimate agree with this result, i.e. when applying clonal interference analysis with.  
For the clonal interference has the correct behavior but the incorrect coefficient, but the application of clonal interference does not give the correct scaling with $L$.  It can be improved if the establishment rate is replaced with the fixation rate (not clear why).  The width of the range around where $v(s)$ is no less than a factor of 2 of is also of order , i.e. specifically between and , which indicates that application of the single $s$ model is valid.

\subsection*{Conclusion}
The focus of this paper is analyzing the dynamics of multiple mutations and their interplay in what is called clonal interference.  

The successional regime occurs when beneficial mutations are rare



%\begin{align*}
%g(n,1,t)={1\over 2+s}(2+s)dt \delta_{n,0} + {1+s\over 2+s}(2+s)dt g(n,2,t-dt)+[1-(2+s)dt] g(n,1,t-dt)
%\end{align*}

\subsection*{Appendix: Notes and Calculations}
by using the average establishment time $\langle \tau_q \rangle$ until it becomes the dominant fitness class.   the mean fitness increases by $s$ in the average establishment time $\langle \tau \rangle$, which allows us to solve for $q$ by tracking the growth of .  Once $n_q(t)$ reaches size $1/qs$ grows deterministically at the rate $(q-1)$.  For very large $t$, $n_q(t)$ is nearly deterministic and can be written as $n_q(t)= (1/qs)e^{qs(t-\tau)}$, where $tau$ is a time dependent stochastic variable.   

the At even larger time scales, such that $st\gg 1$ (FIX THIS), $n_q(t)$ is approximately deterministic.  However, for such large $t$, the stochasticity contributions impact the deterministic growth by altering the time for which it appears that $n_q(t)$ reached size $1/qs$ by some factor $\tau$.  Letting $\tau_q$ be the limit of $\tau$ as $t \rightarrow \infty$, then  In particular, since in the stochastic phase, which is where the primary contribution of stochasticity occurs, growth until the transition in the deterministic phase due to the influx of mutants from the subpopulation $n_{q-1}(t)$, the random variable $tau$ captures the stochasticity such that it will make the later deterministic behavior

Following a simpler birth death analysis describing the fate of a single mutant, the function $n_q(t)$ can be thought of as a composite stochastic processes, whose parts are time-dependent random variables describing the abundances of subpopulations originating from the continual appearance of lineages stemming from single mutants which arise from births in the subpopulation $n_{q-1}(t)$.  

Once this fitness class establishes, it grows exponentially as $n(t)=(1/qs)e^{(q-1)st}$ for a duration of time approximately equal to $ln(Nqs)/[(q-1)s/2]$, at which point it represents a substantial proportion of the total population.  This results in a change of mean fitness equal to  mean fitness has changed by  The fitness class class prior has a selective advantage of $qs$ over the populations mean fitness.

The . there are three parameters that determine the key behavior of adaptations in clonal populations, which are the population size $N$, the beneficial mutation rate $U_b$, and the selective advantage $s$.  What is  

The second is the strong-selection strong-mutation regime and it is referred to as the \textit{concurrent-mutations regime}. In this regime, beneficial mutations occur much more frequently, as is the case in larger populations sizes where $NU \gg 1$.  Selection is assumed to be strong and not dominated by drift.  Consequently, the time between occurrences of beneficial mutations is shorter than the time required for each to fix in the populations.  This results in the a situation where selective sweeps can overlap. This leads to what is know as clonal interference, whereby competing beneficial mutations within the population compete for fixation. Desai and Fisher point out clonal populations are known to operate in this regime, such as in some viral, bacterial and simple eukaryotic populations.  Desai and Fisher's article focuses on the clonal interference which occurs when clonal populations are in the concurrent mutations regime.  

Beneficial mutations are needed to contribute to new genetic variation, while selection eliminates on standing variation.  Desai and Fisher point out the need to understand the interplay between selection and beneficial mutations as they affect the variation in the population, in particular the selection-mutation balance.  Some comments are made on the distinct regime where mutations are weakly beneficial, in which case drift dominates and impedes their fixation through selection.  This regime occurs in small populations, where drift is much stronger than selection, or when beneficial mutations have a very weak effect, such as in the case of synonymous codon usage.  Desai and Fisher choose to focus on the moderate to large populations in strong selection regimes where drift is only important for very small subpopulations.  The figure is discussed and a comparison is drawn between the successional and concurrent mutations regimes.  As mentioned previously, the successional case occurs in small populations sizes or when $U_b$ is small so that beneficial mutations are rare and their occurrences are seperated in time.

In their model, they assume that the population size $N$ is constant.  They denote $U_b$ as the beneficial mutation rate and the fitness increase from each   mutation as $s$ (this is the log fitness).  The speed of evolution $v$ is $d\langle r \rangle / dt$ which is the rate of increase of average fitness in the population.  If the fittest mutant has advantage $ks$ over the mean, then the speed of evolution can be written as the $v=sd\langle k  / dt \rangle$.  

Desai and Fisher proceed with a simple discussion of how mutants establish in the successional regime.  Beneficial mutations occur at total rate of $NU_b$, and a mutant and the probability for it to survive drift is proportional to $s$, provided $1/N<s<1$.  Establishment occurs if the beneficial mutation survives drift, which occurs roughly when the mutant subpopulation reaches size $1/s$.  The reasoning behind this is that for a lineage of size $n$, it takes roughly $n$ generations for drift to change the population size by $n$ individuals, while selection adds $n(1+s)^n-n \approx n^2s$.  Hence if $n^2s>n$, then selection dominates drift, which exactly states that $n>1/s$ is required in order for the mutant lineage to be safe from drift. 

As a result, the mutant lineage will produce beneficial mutations that establish at a total rate of $NU_bs$ per generation, and thus, a new beneficial mutation will establish every $\tau_{establish} = 1/NU_bs$ generations.  Having reached the size $1/s$, the mutant lineage grows almost exponentially at rate $s$, which implies that the time until fixation is given by
\[
N={1 \over s}e^{s \tau_{fix}}.
\]
solving for the fixation time, we obtain $\tau_{fix}=ln(Ns)/s$ generations are required for the beneficial mutation to fix in the population.  
The calculations for the generating function are given as follows. The for
\bibliography{dfreview}
\bibliographystyle{plain}
\end{document}

