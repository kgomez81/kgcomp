\documentclass[12pt, one column]{article}
\usepackage{amsmath}
\usepackage{amssymb}
\usepackage[margin=1.0in]{geometry}


\title{A Review of ``Beneficial Mutation-Selection Balance and the Effect of Linkage on Positive Selection" by Michael M. Desai and Daniel S. Fisher}
\date{Fall 2016}
\author{Kevin Gomez}

\begin{document}
\maketitle
\newpage

%Writing Resources
%https://academicskills.anu.edu.au/node/492

\newpage
\section*{Introduction}
Our understanding of beneficial mutations and their dynamics is comparatively less developed than it is for neutral and deleterious mutations.  The detailed work carried out for mutations of the neutral and deleterious type has provided important insights into genetic diversity in populations, and they have led to extensive developments in our understanding of important evolutionary processes.  However, it is beneficial mutations that are the source of long-term adaptation in species.  Positive selection plays a vital role in defining the evolutionary paths taken by populations, but it can be quite complex in many natural settings.  In such cases, very little of our intuition can help us describe adaptation accurately.  Nonetheless, progress is required in these areas since adaptation remains a key consideration for many of the interesting questions that concern the processes studied in evolutionary biology.  The work of Desai and Fisher provides a significant step in this direction.

In their paper, Desai and Fisher examine adaptation in large and asexually reproducing populations for circumstances in which selection dominates drift and there is no recombination \cite{DesFish07}.  The authors devote the majority of their discussion to the more interesting situation that arises when beneficial mutations occur frequently, and where multiple mutant lineages are present in the population.  In this case, populations are said to be in what Desai and Fisher call the concurrent-mutations regime.  An alternative situation emerges when beneficial mutations are rare, which the authors call the successional regime.  Adaptation in this second regime has very simple behavior, and it is characterized as nothing more than a series of successive sweeps by single mutant lineages.  The dynamics of the successional case are clear.  The concurrent-mutations regime on the other hand is not as well understood, and adaptation in this scenario has only been partially explored in prior work.

There are two important effects that occur in the concurrent-mutations regime, and each influences the behavior of adaptation.  The first of these is clonal interference, and it refers to the situation in which the fate of a mutant lineage arising from a beneficial mutation, or combination of them, is altered by the presence of another superior mutant lineage.  Outcompeted lineages are driven to extinction, and the beneficial mutations that gave rise to them are lost.  The wasted beneficial mutations do not contribute to adaptation, so the speed of evolution increases less and less with increasing rates of beneficial mutations.  The second is the multiple mutations effect, and it refers to the accumulation of multiple beneficial mutations in lineages prior to any one of them sweeping in the population and becoming the dominant subpopulation.  This effect leads to multiple mutant lineages being present in the population, and each will have a distinct number of beneficial mutations.    

Desai and Fisher's analysis of the concurrent-mutations regime assumes all beneficial mutations have identical fitness effects, which has important consequences for the distribution of fitness within the population.  First, by assuming that all beneficial mutations provide the same fitness advantage, any two mutants with an identical number of beneficial mutations will have the same fitness.  In the population, the multiple mutations effect will give rise to many mutant lineages, and these will differ in fitness only when they posses a distinct number of beneficial mutations.  Clonal interference is limited to competition between lineages with distinct numbers of beneficial mutations.  The population can therefore be divided into subpopulations that are each composed of identically fit individuals that all have the same number of beneficial mutations.  These discrete groups are referred to as fitness classes by Desai and Fisher.  As will be discussed in more detail, this framework is extremely useful for developing a detailed description of adaptation in the concurrent-mutation regime, and its usefulness extends beyond the work discussed in this paper.  Moreover, the model that results from their analysis can be used to examine additional situations which include clonal interference effects that are due to beneficial mutations of varying fitness effects.  This review aims to provide an overview of Desai and Fisher's model for the concurrent-mutations regime, their examinations of the effects of deleterious mutations on adaptation, and the importance of the multiple mutations effect in populations that experience clonal interference.

\section*{The Central Ideas Behind the Model}
The heuristic analysis presented by Desai and Fisher yields a clear discussion of the dynamics of adaptation, and it provides an excellent examination of the concepts used in their model of the concurrent-mutations regime.  Among these are the separation between deterministic and stochastic behavior in growth, the timescales key of events in adaptation and their relative sizes in the different regimes, and mutation-selection balance.  We review each of these in order to permit a clearer discussion of the analysis that describes their simplest model.     

The assumptions of the model imply that the growth of each fitness class will either be deterministic or stochastic depending on its size.  In the model, the parameters are the fixed population size $N$, the beneficial mutation rate $U_b$, and the common selective advantage of beneficial mutations $s$.  Since selection dominates drift, meaning $1/s \ll N$, there is a separation of stochastic and deterministic behavior of fitness classes.  For a fitness class with $n$ individuals, the timescale of random drift is on the order of $n$ generations.  If the selective advantage of the fitness class is $s$, then the change in the number of individuals due to selection is approximately $n^2s$.  If $n^2s \ge n$, or equivalently $n>1/s$, then selection dominates.  Otherwise, drift dominates and the fitness classes can be treated stochastically in isolation since $n \ll N$.   

In the successional regime, one extant class generates mutants at an average rate of $N U_b$.  A beneficial mutation is said to establish when the subpopulation with the mutation grows to roughly the size of $1/s$ individuals, at which point selective forces ensure the survival of the mutation.  Only a certain percentage of the beneficial mutations that appear are able to survive stochastic forces and establish.  The probability of fixation gives the fraction that do survive and it is proportional to $s$.  In Desai and Fisher's model this probability is equal to $s$.  The waiting time between appearances of beneficial mutations that are destined to establish is exponentially distributed with mean $1/NU_b s$, and so, the expected time between establishments is $\langle \tau_{est} \rangle = 1/NU_b s$.  Once a beneficial mutation establishes, selection leads to increases in the abundance of the subpopulation carrying the mutation until its size becomes roughly $N$, at which point the mutation is said to have fixed in the population.  During that time, the growth of that subpopulation is approximately exponential with rate $s$, and so its abundance $n(t)$ is roughly equal to $(1/s) e^{st}$.  The amount of time required for a beneficial mutation to fix in the population, $\tau_{fix}$, can be solved for in terms of the parameters of $N$ and $s$ by setting $n(\tau_{fix})=N$, and doing so provides $\tau_{fix} \approx ln(Ns)/s$.  It can be concluded that $\tau_{fix} \ll \tau_{est}$, or rather $N U_b \ll ln(Ns)$, in order for a population to be in the successional regime. 

The concurrent regime occurs when $\tau_{fix} \ge \tau_{est}$, and it is characterized by the presence of multiple distinct fitness classes, which are in direct competition.  The fittest class, referred to as the nose, is the subpopulation that has not yet established and is still subject to stochastic forces.  Once a beneficial mutation establishes, a new nose subsequently forms.  The process repeats and leads to the progressive appearance of fitter classes that eventually establish and introduce variation in fitness into the population.  Selection continuously removes less fit variants, which results in the population's mean fitness increasing over time.  Selection thus drives advancements of the population's mean fitness.  Advancements of the nose are the made possible by mutations and selection.  The rates of advancements are defined as the total gain in fitness divided by the total time between advancements.  Both rates of increase will depend on the fitness gap between the fittest genotype and the population mean fitness, which may be written as the product $qs$.  Beneficial mutation-selection balance is achieved when the two rates are equal, which determine specific values for the fitness gap $qs$ and speed of evolution $v$ at equilibrium. 

Desai and Fisher provide heuristic estimates of the approximate values for both $q$ and $v$, which are roughly equal to those obtained from a more detailed analysis.  In particular, they show that
\begin{equation}
\begin{aligned}
q \sim & {2\ln[Ns] \over \ln[s/U_b]} \\   
\\
v \sim & {2s^2\ln[Ns] \over \ln^2[s/U_b]}
\end{aligned}
\end{equation}
One immediate insight that can be drawn from these expressions relates to the appearance of the $N$ and $U_b$ separately in the expressions $\ln[Ns]$ and $\ln[s/U_b]$, which determine the timescales of selection and mutation, respectively.  As arguments, $Ns$ directly influences the timescales of selection, while $s/U_b$ describes the contribution of selection relative to mutation, to the dynamics of the evolving population.  It is these contributions that set the speed of evolution $v$ in the expression above.   

\section*{Detailed Analysis of the Concurrent Mutations Regime}
The estimates for $q$ and $v$, which result from beneficial mutation-selection balance in asexual populations in the concurrent mutation regime, are confirmed with a detailed analysis of the growth dynamics at the nose and the bulk of the population.  The behavior of each depends on the stochastic variable $\tau_q$ that measures the time between establishment events, which are the moments in which a fitness class at the current nose transitions from stochastic to deterministic behavior.  Carefully derived estimates of the lead $qs$ and the speed of evolution $v$ at equilibrium follow from determining the expected value $\langle \tau_q \rangle$.   

A population in beneficial mutation-selection balance will have a rate of increase at the nose equal to the rate of increase of the mean of the population.  At $t=0$ the fitness class with a $(q-1)s$ selective advantage over the mean of the population, whose abundance is $n_{q-1}(t)$, establishes by reaching size $1/qs$.  Concurrently, a fitter class with selective advantage $qs$ will appear in the population.  Its abundance $n_q(t)$, which is initially zero, will increase only after the fitness class with abundance $n_{q-1}(t)$ has established.  Changes in $n_q(t)$ are initially dominated by stochastic forces because of its small size.  The growth in $n_q(t)$ is due to the appearance of new mutants contributed by the $(q-1)s$ fitness class and their descendants.  The stochastic nature of $n_q(t)$ can be characterized by growth of these mutant lineages that all follow identical branching process.  Estimates for the moments of $n_q(t)$ can be obtained as needed using moment generating functions.

The detailed analysis leading to a careful description of $n_q(t)$ provides a way of measuring the time until establishment for the nose.  For large $t$, the growth of $n_q(t)$ becomes more and more deterministic due to selection.  At these larger timescales the abundance of this fitness class can be written as $n_q(t) = (1/qs) e^{qs(t-\tau)}$, where $\tau$ is a time dependent random variable that accounts for the stochastic contributions which accumulate over time.  In essence, $\tau$ is the offset to time for which the seemingly deterministic growth of $n_q(t)$ at large $t$ appears have began at size $1/qs$ when extrapolating backwards in time.  As $t\rightarrow \infty$, the stochastic contribution becomes less and less significant, and $\tau$ converges to a random variable denoted $\tau_q \equiv \tau (t\rightarrow \infty)$ that is independent of time.  Since $\tau = t +(1/qs)(\ln(1/qs)-\ln(n_q(t)))$, they obtain the moments of $\tau$ from those of $n_q(t)$.  By taking the limit of $\tau$ as $t\rightarrow \infty$, they obtain an expression for $\langle \tau_q \rangle$ below, as well as higher moments. 
\begin{equation} 
\langle \tau_q \rangle = \frac{1}{(q-1)s}\ln\left(\frac{s}{U_b}\frac{(q-1)\sin(\pi/q)}{\pi^{\gamma/q}}\right)
\end{equation}
Steady state occurs when the rates of increase of the population's mean fitness and the nose are equal, and this is achieved for a specific value of $q$.  Equation (2) can be substituted into $n_q(t)$ to track the growth over time of a fitness class beginning at the nose, which then allows us to solve for the steady state $q$.  The mean fitness increases by $s$ in the average establishment time $\langle \tau_q \rangle$.  Once the nose establishes, the corresponding fitness class grows at the rate of $(q-1)s$ until the next establishment, assuming that $q>2$ to ensure that growth does not slow due to saturation.  At that point the mean has increase again by $s$, and the rate of growth is diminished to $(q-2)s$ for another $\langle \tau_q \rangle$ until the next establishment.  This continues until this same class becomes the largest fitness class of size approximately equal to $N$, which occurs in a time span of $(q-1) \langle \tau_q \rangle$.
\begin{equation}
N \approx  \frac{1}{qs} \exp\left[ \frac{q(q-1)s \langle \tau_q \rangle}{2}\right]  
\end{equation}
This is a transcendental equation in $q$, which can be solved using iterations.  A simple zeroth order approximation of $q$ can be used to solve for its dependence on the parameters, and to estimate the rate of evolution $v$. The expressions that follow are similar to those obtained in the heuristic analysis.
\begin{equation}
\begin{aligned}
q \approx & {2\ln[Ns] \over \ln[s/U_b]} \\   
\\
v \approx & s^2 \left[{2\ln[Ns] - \ln[s/U_b] \over \ln^2[s/U_b]}\right]
\end{aligned}
\end{equation} 
The dependence of steady state rate of evolution $v$ on the fitness distribution of the bulk obeys Fisher's fundamental theorem of natural selection.  The growth or decline of fitness classes over a time period of $\tau_q$ is exponential with a rate that depends on how many more (or less) mutations it has above (or below) the mean.  If a particular fitness class has $l$ mutations  more (or less) than the mean, its size will be approximately $N \exp[-\sum_{i=1}^{l} \tau_q] \sim N \exp[(\tau_q/2s)(ls)^2]$.  The variance of this Gaussian fitness distribution, is $\sigma^2 = (s/\tau_q)=v$ which is the rate of evolution.  Furthermore, since the above indicates that $N \exp[(qs)^2/2\sigma^2]\approx (1/qs)$, it follows that $v=\sigma^2 \approx (qs)^2/2\ln(Ns)$.  The magnitude of the lead is the key measure of the fitness distribution's width, and the equality between $v$ and $\sigma^2$ follows specifically from their relationship to the lead.

\section*{The Effect of Deleterious Mutations}
The simple model with a single beneficial $s$ does not incorporate the occurrence of deleterious mutations, which can impact the dependence of $v$ on the mutation rate and population size.  Desai and Fisher discuss adaptation in the case where  deleterious mutations of various fitness effects occur by the results of their simple model.  The magnitudes of these deleterious mutations are denoted $s_d$.  The authors consider two ranges of fitness effects separately where either $s_d \geq s$ or $s_d \ll s$.  They show that in both cases deleterious mutations decrease the mean fitness of the population and reduce the rate at which new mutants establish.  Since the first of these increases the speed of evolution and the second decreases it, the two effects either completely or partially cancel one another depending on the case considered.  They conclude that deleterious mutations do not substantially change the rate of evolution predicted by the simple model.     

In the first case where deleterious fitness effects satisfy $s_d \geq s$, Desai and Fisher argue that the rate of evolution remains the same.  The mutation rate for deleterious mutations in this range is denoted $U_d^{>}$.  The effects of these mutations only need to be considered for the largest subpopulation, since the others are exponentially smaller.  Deleterious mutations lower the mean fitness of the population by $U_d^{>}$ in accordance to the Haldane-Muller principle of mutation load.  Since $s_d \geq s$, mutation-selection balance for deleterious mutations is achieved on a timescale $1/(s_d +U_d^{>}) < 1/s$.  The increase in the lead of the nose by $U_d^{>}$ produces an increase in its growth rate by that amount.  On the other hand, deleterious mutations also occur at the nose at a rate of $U_d^{>}$, which decreases the rate of growth of mutant lineages the nose by that same amount.  The two effects cancel out as a result, and the speed of evolution $v$ remains the same.

In the case of weakly deleterious mutations ($s_d \ll s$) there will be small changes in the speed of evolution $v$.  The mutation rate for weakly deleterious mutations is denoted $U_d^{<}$.  Beginning with the effects at the nose, weakly deleterious mutations will decrease the size of nose advancement from $s$ to $s$ minus an amount that is determined by the mutational load created by deleterious mutations.  The size of the decrease will depend on the size of $U_d^{<}$ relative to $s$.  When $(U_d^{<} / qs)\ln(s/U_b) <<1$ (or roughly $U_d^{<}/s \ll 1$), the rate at which deleterious mutations occur is smaller than the mean rate of establishments occurring at the nose.  The load placed on nose extending mutations reduces the effective $s$ by approximately $(U_d^{<} s_d /(q-1)s)\ln(s/Ub)$, which is much smaller than $s$ given the assumption above.  When $U_d^{<} / s >> 1 $, the rate at which deleterious mutations occur is larger than the mean rate of establishments.  The load placed on nose extending mutations is larger as a result, but it is still on the order of $s_d$ whose size is still significantly smaller than $s$.  In both cases, whether $U_d^{<}/s \ll 1$ or $U_d^{<}/s >> 1$, the effects of deleterious mutations at the nose decreases the rate of growth of mutant lineages by an amount on the order of $s_d$ or less.  In the bulk of the population, deleterious mutations once again lead to a reduction in the mean fitness.  However, since $s_d \ll s$, mutation-selection balance for deleterious mutations in the bulk is achieved on timescales larger than those corresponding to the mean time between establishments $\tau_q$.  The total effect on mean fitness is cumulative over the period of time in which a fitness class at the nose sweeps in the population, i.e. the fitness class becomes the dominant class of the bulk.  This duration of time is roughly equal to $qs/v \approx \ln(s/U_b)/s$.  The reduction to the mean fitness remains similar to the prior description, amounting to a reduction of at most order $U_d^{<}$ and maximized when $1/s_d$ is on the order of the sweep time.  Unfortunately, the precise effect on $v$ from the combined contributions mentioned above cannot be given using this simple analysis.  Additional work has been carried out for this particular scenario in a subsequent paper by Good and Desai \cite{GoodDes14}.

\section*{Clonal Interference and Varying ``s"}
As was mentioned in the introduction, the assumption that all beneficial mutations have the same effect size $s$ eliminates the need to consider clonal interference due to competing beneficial mutations with different fitness effects occurring on the same genetic background.  Generally, beneficial mutations will have different fitness effects, and this has been demonstrated through the use of evolutionary experiments.  Desai and Fisher examine adaptation in the concurrent-mutations regime with varying $s$ for beneficial mutations.  Their discussion does not attempt to provide a rigorous analysis, but rather, it aims to use results of the model to examine what the qualitative effects of varying $s$ are on adaptation.  In particular, the authors are able to point out that in a regime where clonal interference matters, multiple mutations must be accounted for as well.  Below, I'll present a subset of the results derived from the standard clonal interference analysis, and proceed with discussing a specific case examined by Desai and Fisher's that uses the results of the simple model to properly account for the effects of both multiple mutations and clonal interference on adaptation in the clonal-interference regime.

In the previous clonal interference analysis originating from recent work by Gerrish and Lenski (1998), the effect of multiple mutations was ignored and focus was drawn towards competition between single mutations occurring on the same genetic background.  In summary, the main results yields a characteristic fitness effect $s_{CI}$, which estimates the average fitness advantage of beneficial mutations that fix in the population.  The critical value $s_{CI}$ will depend on the population size $N$, the overall beneficial mutation rate $U_b$, and the distribution of fitness effects $\rho(s)$.  From the clonal interference analysis, it can be show that the relationship between them satisfies $-\ln[U_b \rho(s_{CI})] \approx \ln[N s_{CI}]$.  The rate of adaptation is approximately $v_{CI} \sim [s_{CI}(N)]^2$.  Unfortunately, the analysis provides incorrect predictions about the relationship between the beneficial mutation rate $U_b$ and characteristic size $s_{CI}$ of mutations that fix by ignoring multiple mutation effects.  Desai and Fisher demonstrate that multiple mutation effects are relevant in populations that are large enough to experience clonal interference.  A beneficial mutation with fitness effect $s$ will be prevented from fixing in the population if $s/\ln(Ns)$ is much smaller than the establishment rate of a superior mutation.  The rate must be larger than or equal to $NU_b^{>s}s$.  The single-$s$ analysis indicates that multiple mutation effects will occur since $NU_b^{>s} \gg 1/\ln(Ns)$.  So, multiple mutations must be considered with clonal interference when examining the concurrent mutations regime.

The distribution of fitness effects $\rho(s)$ plays an important role in determining how clonal interference and multiple mutations interact and affect adaptation in the concurrent-mutations regime.  A careful analysis of these interactions are examined in subsequent work by Good et al. \cite{Goodetal12}.  Desai and Fisher present arguments which indicate that for a specific class of distributions $\rho(s)$, their single-$s$ model can be used to determine the speed of evolution $v$.  When the distribution of fitness effects is unimodal and decays faster than an exponential, a predominant set of fitness effects $s$ emerges for beneficial mutations, or combination of them, that fix in the population.  Their sizes will all lie within a narrow range of some characteristic fitness advantage $\tilde{s}$.  Desai and Fisher's argument for the existence of such a range begins with considering an initially clonal population.  Mutations of varying sizes appear in either the initial genetic background, or in prior mutant lineages that have already arisen from the initial population.  Double mutants, triple mutants and so on will arise through the effects of multiple mutations.  Eventually a fittest mutant lineage will establish before any other lineage manages to fix in the population, and the extant lineages that are not able to fix will be driven to extinction.  The different fitness advangtages $s$ for the typical mutant lineages that do fix form the predominant range mentioned above.  Mutant lineages with a fitness advantage that is much smaller than $\tilde{s}$, below the predominant range, are routinely outcompeted.  Mutant lineages with a fitness advantage that is much larger than $\tilde{s}$, above the predominant range, are far too rare to affect the fixation of mutations in the predominant range.  Desai and Fisher correctly argue that when the predominant range is narrow enough, the single-$s$ model can be applied using $\tilde{s}$ with $v$ from equation (4) to determine the speed of evolution in the concurrent-mutations regime.  Careful estimates of the characteristic $\tilde{s}$ and the predominant range, corresponding to the type of $\rho(s)$ discussed above, are provided in Good et al. \cite{Goodetal12}, and their calculations for $v$ confirm the arguments given by Desai and Fisher.   

\section*{Conclusion}
Desai and Fisher's analysis of adaptation in the concurrent-mutations regime provides a significant contribution to our understanding evolution.  There are many examples of asexually reproducing populations with linkage that evolve in this regime, which include viruses and many unicellular organisms.  Their model, which is built on key assumptions and approximations, such as the seperation of deterministic and stochastic growth, also yields a reasonably simple description of the dynamics of adaptation and lead to results that can be easily incorporated into future research.  This contrasts with other moment-based approaches in modeling the concurrent-mutations regime that have proven difficult work with.  We have also reviewed Desai and Fisher's examination of the effects of deleterious mutations and varying $s$ on adaptation.  The discussions demonstrate that the single $s$ treatment of beneficial mutations, which may seem unreasonable at first glance, is valid even in many of these situations.  In summary, Desai and Fisher have provided an excellent analysis and a set of results that can be used to characterize adaptation in the concurrent-mutations regime for a variety of situations.

\bibliography{dfreview}
\bibliographystyle{plain}

\end{document}
