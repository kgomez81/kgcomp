\documentclass[12pt, one column]{article}
\usepackage{amsmath}
\usepackage{amssymb}
\usepackage[margin=1.0in]{geometry}


\title{A Review of M. M. Desai's and D. S. Fisher's "Beneficial Mutation-Selection Balance and the Effect of Linkage on Positive Selection"}
\date{Fall 2016}
\author{Kevin Gomez}

\begin{document}
\maketitle
\newpage

%Writing Resources
%https://academicskills.anu.edu.au/node/492

\newpage
\section*{Introduction}
Our understanding of beneficial mutations and their dynamics is comparatively less developed than it is for neutral and deleterious mutations.  The detailed work carried out for the neutral and deleterious mutations has provided important insights into genetic diversity in populations, and they have led to extensive developments in our understanding of evolutionary processes that shape the gene pools of species.  Yet it is beneficial mutations that are the source of long-term adaptation in species.  Positive selection plays an vital role in defining the evolutionary paths taken by populations in many natural situations, but the complexity of adaptation can be formidable.  Nonetheless, progress is required in these areas since adaptation remains an important consideration for many of the key questions one may ask about the processes and observations in evolutionary biology.  The work of Desai and Fisher provides a significant step in this direction.

The primary focus of their discussion is adaptation in asexually reproducing populations with no recombination and selection dominating drift.  In this setting, the behavior of adaptation will depend on the strength of mutation.  When beneficial mutations rarely occur, the dynamics are simple and consist of a series of successive of selective sweeps.  This is called the successional regime. 

Interactions between selection and mutations in large populations lead to different relationships between their parameters and the rate of evolution than those that emerge from the simpler and well-understood successional regime. 

An alternative situation arises in a regime called the concurrent-mutations regime where both selection and mutation are strong.  This particular case is of primary interest to Desai and Fisher, is the focus of the main topics discussed in this review.  When beneficial mutations are produced frequently, they create multiple mutant lineages that are forced to compete with one another.  

The dynamics of adaptation in the concurrent-mutations regime results in wasted beneficial mutations due to competition between mutant lineages.   Adaptation in the concurrent-mutations regime also produces two important effects.  The first of these they call \textit{clonal interference} and it refers to the competition between mutant lineages produced from distinct beneficial mutations occurring in one single lineage.  The use of the term clonal interference in this respect differs from its broader use, but lies in agreement with the recent work beginning with Gerrish and Lenski.  The second is the multiple mutations effect, and it refers to the appearance of mutants with multiple beneficial mutations originating from the same lineage prior to the fixation of the first beneficial mutation arising from that lineage.    

Desai and Fisher's analysis of the concurrent-mutations regime assumes all beneficial mutations have identical fitness effects, which carries important consequences on distribution of fitness within the population.  The only lineages that arise are from the multiple mutations effect since clonal interference cannot occur when fitness effects are all the same.  As a result, the population can be divided into subpopulations that are each composed of identically fit individuals which all have the same number of beneficial mutations.  These discrete groups are referred to as fitness classes by Desai and Fisher.  As will be discussed in more detail, this framework is extremely useful for developing a detailed description of adaptation in the concurrent-mutation regime, and its usefulness extends beyond the work discussed in this paper.  Moreover, the model that results from their analysis, can be used to discuss additional situations which include clonal interference, among other things.  This review aims to provide an overview of Desai and Fisher's model for the concurrent-mutations regime, along with their examinations of the effects of deleterious mutations and clonal interference on adaptation.

\section*{The Central Ideas Behind the Model}
The heuristic analysis presented by Desai and Fisher yields a clear discussion of the dynamics of adaptation, and it provides an excellent examination of the concepts used in their model of the concurrent-mutations regime.  Among these are the separation between deterministic and stochastic behavior in growth, the time scales key events in adaptation and their relative sizes in the different regimes, and mutation-selection balance.  We review each of these in order to permit a clearer discussion of the analysis that describes their simplest model.     

The assumptions of the model imply that the growth of each fitness class will either be deterministic or stochastic depending on its size.  In the model, the parameters are the fixed population size $N$, the beneficial mutation rate $U_b$, and the common selective advantage of beneficial mutations $s$.  Since selection dominates drift, meaning $1/s \ll N$, a separation of stochastic and deterministic behavior of fitness classes .  For a fitness class of size $n$, it takes random drift on the order of $n$ generations to decrease its size by $n$.  If the selective advantage of the fitness class is $s$, then in $n$ generations its size will increase on average by approximately $n^2s$.  If $n^2s \ge n$, or equivalently $n>1/s$, then selection dominates.  Otherwise, drift dominates and the fitness classes can be treated stochastically in isolation since $n \ll N$.  

Populations evolve in the successional regime when specific conditions are met that involve the parameters $N$, $U_b$ and $s$, or will otherwise be in concurrent-mutations regime.  In the successional regime, one ruling class generates mutants at an average rate of $N U_b$.  A beneficial mutation is said to establish when the subpopulation with the mutation grows to roughly the size of $1/s$, at which point selective forces ensure the survival of the mutation.  Only a certain percentage of the beneficial mutations that appear are able to survive stochastic forces and establish.  The probability of fixation gives the fraction that do survive and it is proportional to $s$.  In Desai and Fisher's model this probability is equal to $s$.  The waiting time between appearances of beneficial mutations that are destined to establish is exponentially distributed with mean $1/NU_b s$, and so, the expected time between establishments is $\langle \tau_{est} \rangle = 1/NU_b s$.  Once a beneficial mutation establishes, selection lead to increases in the abundance of the subpopulation carrying the mutation until its size becomes roughly $N$, at which point the mutation is said to have fixed in the population.  During that time, the growth of that subpopulation is approximately exponential with rate $s$, and so its abundance $n(t)$ is roughly equal to $(1/s) e^{st}$.  The amount of time required for a beneficial mutation to fix in the population, $\tau_{fix}$, can be solved for in terms of the parameters of $N$ and $s$ by setting $n(tau_{fix})=N$, and doing so provides $\tau_{fix} \approx ln(Ns)/s$.  It can be concluded that $\tau_{fix} \ll \tau_{est}$, or rather $N U_b \ll ln(Ns)$, in order for a population to be in the successional regime. 

The concurrent regime occurs when $\tau_{fix} \ge \tau_{est}$, and it is characterized by the presence of multiple distinct fitness classes, which are in direct competition.  The fittest class, referred to as the nose, is the subpopulation that has not yet established and is still subject to stochastic forces.  Once a beneficial mutation establishes, a new nose subsequently forms.  The process repeats and leads to the progressive appearance of fitter classes that eventually establish and introduce variation in fitness into the population.  Selection on the other hand continuously removes this variation, which results in the population's mean fitness increasing over time.  Selection thus drives advancements of the population's mean fitness, while mutations are responsible for advancements of the nose.  These rate of these advancements are measured by the rate of increase, denoted $v$, and they are defined as the total gain in fitness divided by the associated mean time between advancements.  Both rates of increase will depend on the fitness gap between the fitness of the fittest fitness class and the populations mean fitness, which may be written as the product $qs$; increases in $qs$ induce increases in each of them.  Mutation-selection balance is achieved when the two are equal, which determine specific values for the associated fitness gap $qs$ and speed of evolution $v$ at equilibrium. 

Desai and Fisher provide heuristic estimates of the approximate values for both $q$ and $v$, which are roughly equal to those obtained from a more detailed analysis.  In particular, they show that
\begin{equation}
\begin{aligned}
q \sim & {2ln[Ns] \over ln[s/U_b]} \\   
\\
v \sim & {2s^2ln[Ns] \over ln^2[s/U_b]}
\end{aligned}
\end{equation}
One immediate insight that can be drawn from these expressions relates to the appearance of the $N$ and $U_b$ separately in the parameters $ln[Ns]$ and $ln[s/U_b]$, which determine the timescales of selection and mutation, respectively.  As arguments, $Ns$ directly influences the timescales of selection, while $s/U_b$ describes the respective contributions of selection and mutation to dynamics of the evolving population.  It is these contributions that set the speed of evolution $v$ in the expression above.   

\section*{Detailed Analysis of the Concurrent Mutations Regime}
The estimates for $q$ and $v$, which result from mutation-selection balance in asexual populations following the concurrent mutation regime, are confirmed with a detailed analysis of the growth dynamics at the nose and the bulk of the population.  The behavior of each depends on the stochastic variable $\tau_q$ which measures the time between the appearance of a new fittest class and its establishment.  Carefully derived estimates of the lead $qs$ and the speed of evolution $v$ at equilibrium follow from determining the expected value $\langle \tau_q \rangle$.   

As previously mentioned, a population in mutation-selection balance will have a rate of increase at the nose equal to the rate of increase of the mean of the population.  At $t=0$ the fitness class with a $(q-1)s$ selective advantage over the mean of the population, whose abundance is $n_{q-1}(t)$, establishes by reaching size $1/qs$.  Concurrently, a fitter class with selective advantage $qs$ and abundance $n_q(t)$ appears in the population.  Changes in $n_q(t)$ are initially dominated by stochastic forces due to its small size, and its growth is due to the appearance of new mutants and their descendants contributed by the $(q-1)s$ fitness class. Consequently, the stochastic nature of $n_q(t)$ is characterized by growth of these mutant lineages that all follow identical branching process.  Using moment generating functions, estimates for the moments of $n_q(t)$ can be obtained as needed.

The detailed analysis leading to a careful description of $n_q(t)$ provides a way of measuring the time until establishment for the nose.  For large $t$, the growth of $n_q(t)$ becomes more and more deterministic due to selection.  At these larger time scales the abundance of this fitness class can be written as $n_q(t) = (1/qs) e^{qs(t-\tau)}$, where $\tau$ is a time dependent random variable that accounts for the stochastic contributions which accumulate over time.  In essence, $\tau$ is the offset to time for which the seemingly deterministic growth of $n_q(t)$ at large $t$ appears have reached size $1/qs$ by extrapolating backwards in time.  As $t\rightarrow \infty$, the stochastic contribution becomes less and less significant, and $\tau$ converges to a random variable denoted $\tau_q \equiv \tau (t\rightarrow \infty)$ that is independent of time.  Since $\tau = t +(1/qs)(ln(1/qs)-ln(n_q(t)))$, we can obtain the moments of $\tau$ from those of $n_q(t)$.  By taking the limit of $\tau$ as $t\rightarrow \infty$, we obtain an expression for $\langle \tau_q \rangle$ below, as well as higher moments. 
\begin{equation} 
\langle \tau_q \rangle = \frac{1}{(q-1)s}ln\left(\frac{s}{U_b}\frac{(q-1)sin(\pi/q)}{\pi^{\gamma/q}}\right)
\end{equation}
Steady state occurs when the rates of increase of the population's mean fitness and the nose are equal, and this is achieved for a specific value of $q$.  Equation (2) can be used to track the growth over time of a fitness class beginning at the nose, which then allows us to solve for the steady state $q$.  The mean fitness increases by $s$ in the average establishment time $\langle \tau_q \rangle$.  Once the nose establishes, the corresponding fitness class grows at the rate of $(q-1)s$ until the next establishment, assuming that $q>2$ to ensure that growth does not slow due to saturation.  At that point the mean has increase again by $s$, and the rate of growth is diminished to $(q-2)s$ for another $\langle \tau_q \rangle$ until the next establishment.  This continues until this same class becomes the largest fitness class of size approximately equal to $N$, which occurs in a time span of $(q-1) \langle \tau_q \rangle$.
\begin{equation}
N \approx  \frac{1}{qs} \exp\left[ \frac{q(q-1)s \langle \tau_q \rangle}{2}\right]  
\end{equation}
In this expression we may substitute in for $\langle \tau_q \rangle$ to obtain an transcendental equation in $q$, which can be solved using iterations.  A simple zeroth order approximation of $q$ can be used to solve for its dependence on the parameters, and an estimate of rate of evolution $v$. The expressions that follow are identical to those obtained in the heuristic analysis.
\begin{equation}
\begin{aligned}
q \approx & {2ln[Ns] \over ln[s/U_b]} \\   
\\
v \approx & s^2 \left[{2ln[Ns] - ln[s/U_b] \over ln^2[s/U_b]}\right]
\end{aligned}
\end{equation} 
The resulting steady state rate of evolution $v$ bears a relationship to the fitness distribution of the bulk, that follows the Fisher's fundamental theorem of natural selection.  The growth and decline of a fitness classes over a time period of $\tau_q$, follow the description above, and hence its growth (or decline) is exponential with a rate dependent that depends on how many more (or less) mutations it has above (or below) the mean.  If a particular fitness class has $l$ mutations  more (or less) than the mean, its size will be approximately $N \exp[-\sum_{i=1}^{l} \tau_q] \sim N \exp[(\tau_q/2s)(ls)^2]$.  The variance of this Gaussian fitness distribution, is $\sigma^2 = (s/\tau_q)=v$ which is the rate of evolution.  Furthermore, since the above indicates that $N \exp[(qs)^2/2\sigma^2]\approx (1/qs)$, it follows that $v=\sigma^2 \approx (qs)^2/2ln(Ns)$.  The magnitude of the lead is the key measure of the fitness distribution's width, and the equality between $v$ and $\sigma^2$ follows specifically from their relationship to the lead.

\section*{The Effect of Deleterious Mutations}
The simple model with a single $s$ does not incorporate the occurrence of deleterious mutations, which can impact the dependence of $v$ on the mutation rate and population size.  In their analysis, deleterious mutations with effects $s_d$ that vary in size are considered in the framework of the simple model with a single $s$ model.  Two ranges of fitness effects are examined, but in both cases, the contributions to the dynamics of the bulk and the nose act in opposition with respect to the changes they induce on their rates of increase.   

In the first range of deleterious effects corresponding to $s_d \geq s$, with mutation rate $U_d^{>}$, a qualitative analysis reveals the fitness distribution of the bulk changes, but the steady state $v$ remains unaffected.  Deleterious mutations lower the mean fitness of the population by $U_d^{>}$ through its effects on the largest subpopulation.  Since $s_d \geq s$, mutation-selection balance for deleterious mutations is achieved on a time scale $1/(s_d +U_d^{>}) < 1/s$.  At this equilibrium, the frequency of deleterious mutants is $U_d^{>}/s_d$ with relative fitness $0$ and the unaffected portion is $(1-U_d^{>}/s_d)$ with relative fitness $-s_d$, and hence the new mean is $-s_d (U_d^{>}/s_d) = -U_d^{>}$.  This increases the lead of the nose by that amount, and consequently its growth by the same amount.  At the nose, deleterious mutations occurring at rate $U_d^{>}$ simply decrease the rate of growth of mutant lineages the nose by that amount.  The two effect cancel out as a result, and the speed of evolution $v$ remains mostly unchanged.

Weakly deleterious mutations, whose mutation rate is denoted rate $U_d^{<}$, and corresponds to the case where $s_d \ll s$, affect the bulk and nose  in ways that require more careful examination.  Deleterious mutations decrease the gain in fitness to $s$ minus an amount corresponding to the mutational load which depends on the magnitude of $U_d^{<}$ relative to $s_d$.  When $(U_d^{<} / qs)ln(s/U_b) <<1$ (or equivalently $U_d^{<}/s \ll 1$), the rate of deleterious mutations is smaller than the the rate of establishments occurring at the nose.  The associated load placed on nose extending mutations is small, and is approximately $(U_d^{<} sd /(q-1)s)ln(s/Ub)$.  In this case, the effective $s$ is reduced on average by that amount, and it is significantly smaller than $s$.  When $U_d^{<} / s >> 1 $, deleterious mutations occur at a faster rate than establishments on average.  The load placed on nose extending mutations is larger as a result, but it is still on the order of $s_d$.  In both cases, the  effects of deleterious mutations on the bulk lead a reduction in the mean fitness.  However, since $s> s_d$, mutation-selection balance is achieved on time scales larger than those corresponding to the mean establishment time.  As a result, the total effect on mean fitness is cumulative over period of the sweep for which a fitness class at the nose becomes the dominant class of the bulk, roughly $qs/v \approx ln(s/U_b)/s$.  The reduction to the mean fitness remains similar to the prior description, amounting to a reduction of at most order $U_d^{<}$ and maximized when $1/s_d$ is on the order of the sweep time.  Unfortunately, the precise effect on $v$ from the combined contributions mentioned above cannot be given with this using this simple analysis.  This may seem like an unsatisfactory analysis, but additional work has been carried out for this particular scenario.  This is examined in further work by Good and Desai, and the results indicate that the first analysis was correct, while for the second case where $s_d \ll s$, the more detailed analysis confirms an identical result.

\section*{Clonal Interference and Varying "s"}
As was mentioned in the introduction, the assumption that all beneficial mutations have the same effect size $s$ eliminates the effect of clonal interference in the simple model.  Generally, beneficial mutations do have varying fitness effects, and in fact, a large body of experimental work exists devoted to understanding what the distribution of fitness effects are.  For this reason, it is important to at least consider how adaptation in the concurrent-mutations regime might change when fitness effects vary.  Desai and Fisher do not aim to provide a rigorous analysis of the concurrent-mutations regime with a relaxation of the fixed $s$, that work is left to subsequent papers.  However, the authors do show that the single $s$ model .

In the concurrent-mutations regime with varying fitness effects, both clonal interference and multiple mutations effects lead to a waste of beneficial mutations as previously described.  In the standard clonal interference analysis, initially stemming from recent work by Gerrish and Lenski, the effect of multiple mutations is ignored and focus is drawn towards the competition between single mutations occurring on the same genetic background.  In summary, clonal interference yields an characteristic fitness effect $s_{CI}$ that depends on the population size $N$, which will determine the rate at which adaptation progresses $v_{CI} \sim [s_{CI}]^2$.  The relationship between $s_{CI}$ and population size $N$ will depend on the distribution of selective advantages $\rho(s)$, and it is approximately given by $-\ln[U_b p(s_{CI})] \approx \ln[N s_{CI}]$.  The analysis provides incorrect predictions about the relationship between the beneficial mutation rate $U_b$ and characteristic size $s_{CI}$ of mutations that fix, and this is due to the discounting of multiple mutations.  As Desai and Fisher demonstrate in their paper, multiple mutations are relevant in populations that are large enough to experience clonal interference.  Clonal interference impeding the fixation of a beneficial mutation of size $s$ will occur when the establishment rate of mutations superior to it, which is greater then or equal to $NU_b^{>s}s$, exceeds the fixation time of a mutations of size $s$, $s/\ln(Ns)$.  The results of the single $s$ model that multiple mutations occur since $NU_b^{>s} \gg 1/\ln(Ns)$, so both effects play a role in determining the behavior of adaptation in the concurrent mutations regime with varying $s$.

When clonal interference and multiple mutations interact, a predominant set of fitness effects emerges whose sizes will all be roughly in the range of some $\tilde{s}$ occurring with a mutation rate of $\tilde{U}_b$.  


In an initially clonal population, mutations of varying sizes will appear in either the initial genetic background, or in prior mutant lineages that have already arisen from the initial population. Double mutants, triple mutants and so on will appear through the effects of multiple mutations.  Eventually, a fittest mutant will give rise to a lineage that establishes before any others manage to fix in the population, driving them to extinction.  Similar cycles contineu 


The typical fitness effect sizes that produce such predominant mutations, will depend on the population size, the beneficial mutation rate, and the distribution of selective advantages.  

The value of $\tilde{s}$ cannot be determined as readily as $s_{CI}$.  However, the existence of such a range near some value $\tilde{s}$ follows from the fact that mutations much smaller than $\tilde{s}$ are routinely outcompeted, while those significantly greater than $\tilde{s}$ are much too rare by definition to prevent the fixation of mutations in the predominant range.  Both $\tilde{s}$, and more importantly the range of the range of around it, and $\tilde{U}_b$, will depend on the the parameters $N$, $U_b$, and the distribution of selective advantages $\rho(s)$.  Unlike the characteristic selective advantage $s_{CI}$ obtained in the clonal interference analysis, the rate of increase in $\tilde{s}$ due to increases in $N$ and $U_b$ will be less.

Given a range of predominant mutations that is sufficiently narrow, the speed of evolution $v(s)$ from the single $s$ model can be used to determine the predominant $s$, $\tilde{s}$, along with an approximation for $v$.  This can be achieved by using the mutation rate $U_b$ typical to the predominant range, and calculating $\tilde{s}$ with corresponding $v$ from equation (5).
\begin{equation}
v = \max_{s} v(s)=v(\tilde{s})
\end{equation}

The result can be checked for consistency, by prescribing a functional form for the mutation rate of $s$ given by $\mu(s)=U_b \rho(s)$, and examining the two cases for the shape of $\mu(s)$.  The function form of $\mu(s)$ is taken as the function $\mu(s) = \exp[-\ell - (s/\sigma)^\beta]$, where $U_b \propto \exp[-\ell]$ so that $\ell$ provides a parameter for mutations, $\sigma$ is the characteristic selective advantage, and $\beta $ prescribes the size of the distributions tail.  The additional parameters $L=\ln(N\sigma)$ and $\Lambda(s)= \ln(1/\mu(s))$ can be defined.  In the case where $\beta > 1$, the tail of $\rho(s)$ is short.  When the population size is sufficiently large and $2L/\Lambda(s)\gg 1$, the predominant $s$ approximation yields equation (6) for $\tilde{s}$ and $v$ using the large $q$ result from the single $s$ model.
\begin{equation}
\begin{aligned}
\tilde{s} &= \sigma\left[ \frac{\ell}{\beta-1} \right]^{1/\beta}\\
\\
v & \approx C_\beta \sigma^2 \frac{2 \ln(N\sigma)}{\ell^{2-2\beta}}
\end{aligned}
\end{equation}
Solving for the value of $q$ provides that it is indeed in the range used to obtain the estimate, mainly that $q=2L(\beta-1)/\ell \beta$ is fairly large as expected.  The expression for $v$ in (6) shows that  .  

Moreover, the where $v(s)$ is obtained from the velocity estimations from the single $s$ model.  Short-tailed for which leads to an dependence on $N$ and $U_b$ which is determined by  the size of, but can provide estimates.  When, we should obtain a “$q$” that allows for the application of (41), which provides an estimate  from maximizing $v(s)$ providing (58), and gives s that is roughly independent of $N$ and $U_b$.  We also get that does not depend strongly on $N$, but it does increase as $U_b$ increases, following from the fact that $U_b$ leads to more multiple mutations become more important than single large mutations, which requires to grow.  This leads to identical dependence on $N$ but weaker dependence on $U_b$ in $v(s)$.  Not so large $L=ln(Ns)$, will lead to more complicated behavior due to some s crossing over into regime where so large $q$ results becomes inconsistent.  In the large limit $2L/\ell$ the estimates for , , and , indicates that $v(s)$ grows faster than linearly in $ln(N)$, but the dependence on $U_b$ is negligible, but does determine how large $N$ must be, while $q$ is never really large for long-tailed because  increases substantially with $N$.  In short tailed case, multiple mutations matter and contribute.  When, $q<2$ (but sticks to 2), and leads to successional behavior for which which agrees with estimate of $v(s)$ obtained for, and for this range, the clonal-interference estimate agree with this result, i.e. when applying clonal interference analysis with.  
For the clonal interference has the correct behavior but the incorrect coefficient, but the application of clonal interference does not give the correct scaling with $L$.  It can be improved if the establishment rate is replaced with the fixation rate (not clear why).  The width of the range around where $v(s)$ is no less than a factor of 2 of is also of order , i.e. specifically between and , which indicates that application of the single $s$ model is valid.

\section*{Conclusion}
Desai and Fisher's analysis of adaptation in a in the concurrent-mutations regime for asexually reproducing populations with likage provides a significant contribution to our understanding of evolution, occurrences of small effect mutations leads to additional wasted beneficial mutations through clonal interference, where in these small effect beneficial mutations are outcompeted by large effect ones.  The application of Garrish and Lenski's (1998) analysis which derives a minimal effect size sci which fixes in the wild type, and leads to a successional mutation behavior.  The dependence of sci on the population size N following Garrish and Lenski's analysis provides $ln[1/U_b p(s_{CI})]\approx ln[Ns_{CI}]$ ($p(s)$ is distribution of fitness effects “$s$”), which makes a crucial and incorrect assumption in which double mutants do not occur, and results in incorrect predictions. Populations large enough for clonal interference to matter are also large enough for bound mutants to repeatedly appear.  Clonal interference can affect the fixation of a mutation of size s only when the establishment rate of  mutations stronger than $s$, which is atleast $NU_b^{>s} \gg 1/ln(Ns)$ indicating that clonal interference will play a role, then so too will mutltiple mutations (above is roughly the condition for concurrent mutations = multiple mutations). 

\bibliography{dfreview}
\bibliographystyle{plain}
\end{document}
